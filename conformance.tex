\specchapter{Conformance}

\specitem
"Shall" is interpreted as a requirement imposed on the implementation or
program; and "shall not" is interpreted as a prohibition.

\specitem
"May" is used to clarify that a particular interpretation of a requirement of
this specification is considered within the acceptable bounds for conformance.
Conversely, "may not" is used to denote an interpretation which is not
considered conformant.

\specitem
A \textbf{strictly conforming implementation} shall meet the following
requirements:

\specsubitem
It shall implement all of the behavior defined in the authoritative text of this
\\ specification.

\specsubitem
It shall not implement any behavior which is \textbf{included} by
\itemref{Scope}{1} but is not defined by this specification.

\informative{This is to say that vendor extensions are prohibited of conformant
implementations.}

\specsubitem
It may implement behavior which is \textbf{excluded} by \itemref{Scope}{2} and
which is not defined by this specification.

\specitem
A \textbf{conforming freestanding implementation} shall implement all
requirements of this specification \textit{except} for those defined in
\chref{Runtime Library}. A \textbf{conforming hosted implementation} shall
implement all requirements of this specification, including those defined in
\chref{Runtime Library}.

\informative{Some language features require the implementation to provide the
features defined by \chref{Runtime Library}. In order to use a conformant
freestanding implementation, the program may be required to provide its own
implementation of the features defined by \chref{Runtime Library}; or refrain
from using language features which require these implementations.}
