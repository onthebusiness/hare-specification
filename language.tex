\specchapter{Language}

\specsection{Notation}

\informative{A summary of the language syntax is given in \appxref{Language
syntax summary}.}

\specsubitem
The notation used in this specification indicates non-terminals with
\textit{italic type}, terminals with \terminal{bold type}, and optional
symbols use "opt" in subscript. Non-terminals referenced in the text use the
\under{expression} notation. The following example denotes an optional
\nonterminal{expression} enclosed in literal braces:

\begin{grammar}
\terminal{\{} \optional{\nonterminal{expression}} \terminal{\}}
\end{grammar}

\specsubitem
When there are multiple options for a single non-terminal, they will either be
printed on successive lines, or the preceeding authoritative text shall use the
key phrase "one of".

\specsubitem
A non-terminal is defined with its name, a colon (':'), and the options;
indented and shown with one option per line. For example,
\nonterminal{switch-cases} is defined like so:

\begin{grammar}
\nonterminaldef{switch-cases}\\
\nonterminal{switch-case}\\
\nonterminal{switch-case} \terminal{,}\\
\nonterminal{switch-case} \terminal{,} \nonterminal{switch-cases}
\end{grammar}

\specsection{Lexical analysis}

\begin{grammar}
\nonterminaldef{token}\\
	\nonterminal{comment} \\
	\nonterminal{integer-literal} \\
	\nonterminal{floating-literal} \\
	\nonterminal{rune-literal} \\
	\nonterminal{string-section} \\
	\nonterminal{keyword} \\
	\nonterminal{name} \\
	\nonterminal{operator} \\
	\nonterminal{attribute} \\
	\nonterminal{invalid-attribute}

\nonterminaldef{operator}\oneof \\
	\terminal{!}
	\terminal{!=}
	\terminal{\%}
	\terminal{\%=}
	\terminal{\&}
	\terminal{\&\&}
	\terminal{\&\&=}
	\terminal{\&=}
	\terminal{(}
	\terminal{)}
	\terminal{*}
	\terminal{*=}
	\terminal{+}
	\terminal{+=}
	\terminal{,}
	\terminal{-}
	\terminal{-=}
	\terminal{.}
	\terminal{..}
	\terminal{...}
	\terminal{/}
	\terminal{/=}
	\terminal{:}
	\terminal{::}
	\terminal{;}
	\terminal{<}
	\terminal{\textless{}\textless{}}
	\terminal{\textless{}\textless{}=}
	\terminal{<=}
	\terminal{=}
	\terminal{==}
	\terminal{=>}
	\terminal{>}
	\terminal{>=}
	\terminal{\textgreater{}\textgreater{}}
	\terminal{\textgreater{}\textgreater{}=}
	\terminal{?}
	\terminal{[}
	\terminal{]}
	\terminal{\textasciicircum}
	\terminal{\textasciicircum=}
	\terminal{\textasciicircum\textasciicircum}
	\terminal{\textasciicircum\textasciicircum=}
	\terminal{\{}
	\terminal{|}
	\terminal{|=}
	\terminal{||}
	\terminal{||=}
	\terminal{\}}
	\terminal{\textasciitilde}

\nonterminaldef{comment}\exactly\\
	\terminal{//} \nonterminal{comment-chars}

\nonterminaldef{comment-chars}\exactly\\
	\nonterminal{comment-char} \optional{\nonterminal{comment-chars}}

\nonterminaldef{comment-char}\\
	\norm{Any character other than \code{U+000A}}

\end{grammar}

\specsubitem
A token is the smallest unit of meaning in the Hare grammar. The lexical
analysis phase processes a source file to produce a stream of tokens by matching
the terminals with the input text.

\specsubitem
Tokens may be separated by \textit{white-space} characters, which are defined as
the Unicode codepoints \code{U+0009} (horizontal tabulation), \code{U+000A}
(line feed), and \code{U+0020} (space). Any number of white-space characters may
be inserted between tokens, either to disambiguate from subsequent tokens, or
for aesthetic purposes. This white-space is discarded during the lexical
analysis phase.

\informative{Within a single token, white-space is meaningful. For example, the
\nonterminal{string-literal} token is defined by two quotation marks
\terminal{"} enclosing any number of literal characters. The enclosed
characters are considered part of the \nonterminal{string-literal} token and
any white-space therein is not discarded.}

\specsubitem
The lexical analysis process shall repeatedly consume Unicode characters from
the source file input until there are no more characters to consume. White-space
characters shall be discarded. When a non-white-space character is encountered,
it shall mark the beginning of a token: the longest sequence of characters which
constitutes a token shall then be consumed and emitted to the token stream,
unless the token is a \nonterminal{comment}, in which case it shall be
discarded. If no token can be formed, a diagnostic message shall be printed and
the translation phase shall abort.

\specsection{Keywords}

\begin{grammar}
\nonterminaldef{keyword}\oneof \\
\terminal{alloc}
\terminal{assert}
\terminal{as}
\terminal{bool}
\terminal{break}
\terminal{char}
\terminal{const}
\terminal{continue}
\terminal{def}
\terminal{defer}
\terminal{else}
\terminal{enum}
\terminal{export}
\terminal{f32}
\terminal{f64}
\terminal{false}
\terminal{fn}
\terminal{for}
\terminal{free}
\terminal{i16}
\terminal{i32}
\terminal{i64}
\terminal{i8}
\terminal{if}
\terminal{int}
\terminal{is}
\terminal{len}
\terminal{let}
\terminal{match}
\terminal{nullable}
\terminal{null}
\terminal{offset}
\terminal{return}
\terminal{size}
\terminal{static}
\terminal{struct}
\terminal{str}
\terminal{switch}
\terminal{true}
\terminal{u16}
\terminal{u32}
\terminal{u64}
\terminal{u8}
\terminal{uintptr}
\terminal{uint}
\terminal{union}
\terminal{use}
\terminal{void}
\terminal{while}
\end{grammar}

\specsubitem
Keywords (or \textit{reserved words}) are terminals with special meaning. These
names are case-sensitive. Keywords are reserved for elements of the syntax
and shall not appear in user-defined names, in particular in
\secref{Identifiers}.

\specsection{Identifiers}

\begin{grammar}
\nonterminaldef{identifier}\\
\nonterminal{name}\\
\nonterminal{name} \terminal{::} \nonterminal{identifier}\\

\nonterminaldef{name}\\
\nonterminal{nondigit}\\
\nonterminal{name} \nonterminal{alnum}\\

\nonterminaldef{nondigit}\oneof \\
\terminal{a} \terminal{b} \terminal{c} \terminal{d} \terminal{e} \terminal{f}
\terminal{g} \terminal{h} \terminal{i} \terminal{j} \terminal{k} \terminal{l}
\terminal{m} \\
\terminal{n} \terminal{o} \terminal{p} \terminal{q} \terminal{r}
\terminal{s} \terminal{t} \terminal{u} \terminal{v} \terminal{w} \terminal{x}
\terminal{y} \terminal{z} \\
\terminal{A} \terminal{B} \terminal{C} \terminal{D} \terminal{E} \terminal{F}
\terminal{G} \terminal{H} \terminal{I} \terminal{J} \terminal{K} \terminal{L}
\terminal{M} \\
\terminal{N} \terminal{O} \terminal{P} \terminal{Q} \terminal{R}
\terminal{S} \terminal{T} \terminal{U} \terminal{V} \terminal{W} \terminal{X}
\terminal{Y} \terminal{Z} \\
\terminal{\_}\\

\nonterminaldef{digit}\oneof \\
\terminal{0} \terminal{1} \terminal{2} \terminal{3} \terminal{4} \terminal{5}
\terminal{6} \terminal{7} \terminal{8} \terminal{9}\\

\nonterminaldef{alnum}\\
\nonterminal{digit} \\
\nonterminal{nondigit} \\
\end{grammar}

% TODO: The role of the namespace part could be described better

\specsubitem
An \textit{identifier} is a user-defined name which denotes a module, object,
function, type alias, struct or union member, or enumeration member.

\specsubitem
An identifier is only meaningful within a specific \textit{scope} of the
program. The scope is defined by the region of the AST in which the identifier
is applicable; it may be the program, a translation unit, a sub-unit, a
function, or an \nonterminal{expression-list}. The identifier is considered
\textit{visible} within the region that defines its scope.

\specsubitem
A translation unit is assigned a unique \textit{namespace} within the program.
These namespaces may be nested recursively; that is to say that a translation
unit may have a \textit{parent} which is another translation unit. One
translation unit may be assigned to the \textit{root namespace}, which has no
name.

\specsubitem
Identifiers declared within a translation unit scope are assigned the namespace
of the translation unit. The double-colon terminal \terminal{::} is used to
denote the namespace of an identifier, ordered from least to most specific.

\specsubitem
An identifier is either \textit{fully-qualified} or \textit{unqualified}.
Unqualified identifiers require the context of their enclosing scope to be
interpreted unambiguously. Fully-qualified identifiers are used for
\textit{exported} identifiers, and include the namespace in which they reside.

\example{The fully-qualified identifier \code{sys::start::start\_ha} qualifies
the un-qualified identifier \code{start\_ha} with the start namespace, which is
itself a member of the sys namespace.}

\specsubitem
An identifier without the namespace qualification may be fully-qualified
regardless, if it exists in the root namespace.

\specsubitem
The implementation may define the maximum length of a \nonterminal{identifier}
or \nonterminal{name}.

% TODO: constants
% TODO: operators
% TODO: expressions
\specsection{Declarations}

% TODO: split grammars across pagebreaks
\newpage
\begin{grammar}
\nonterminaldef{declarations} \\
	\optional{\terminal{export}} \nonterminal{declaration} \terminal{;} \\
	\optional{\terminal{export}} \nonterminal{declaration} \terminal{;} \nonterminal{declarations} \\

\nonterminaldef{declaration} \\
	\nonterminal{global-declaration} \\
	\nonterminal{constant-declaration} \\
	\nonterminal{type-declaration} \\
	\nonterminal{function-declaration} \\
\end{grammar}

A \nonterminal{declaration} specifies the interpretation and attributes of a set
of \nonterminal{identifier}s.

\specsubitem
The \nonterminal{identifier}s shall be visible anywhere within the current
translation unit. If the \terminal{export} keyword is used, the
\nonterminal{identifier}s shall be part of the unit's exported interface.

\specsubsection{Global declarations}

\begin{grammar}
\nonterminaldef{global-declaration} \\
	\terminal{let} \nonterminal{global-bindings} \\
	\terminal{const} \nonterminal{global-bindings} \\

\nonterminaldef{global-bindings} \\
	\nonterminal{global-binding} \optional{\terminal{,}} \\
	\nonterminal{global-bindings} \terminal{,} \nonterminal{global-binding} \\

\nonterminaldef{global-binding} \\
	\optional{\nonterminal{decl-attr}} \nonterminal{identifier} \terminal{:} \nonterminal{type} \terminal{=} \nonterminal{simple-expression} \\

\nonterminaldef{decl-attr} \\
	\terminal{@symbol} \terminal{(} \nonterminal{string-constant} \terminal{)}
\end{grammar}

\specsubsubitem
In a \nonterminal{global-declaration}, sufficient space shall be reserved for
each \nonterminal{identifier} in the \nonterminal{global-bindings} to store
the \nonterminal{type} associated with it. That storage shall be initialized to
the value of the \nonterminal{simple-expression} and shall have alignment
greater than or equal to the necessary alignment for the \nonterminal{type}. In
the \terminal{const} form, the \nonterminal{type}s shall have the constant flag
enabled by default.

\specsubsubitem
A \nonterminal{global-binding}'s \nonterminal{simple-expression} shall be
limited to the \secref{Translation compatible expression subset}, and shall be
evaluated in the translation environment. The type of the value of the
\nonterminal{simple-expression} shall assignable to \nonterminal{type}.

\specsubsubitem
The interpretation of the \terminal{@symbol} form of \nonterminal{decl-attr} is
implementation-defined.

\specsubsection{Constant declarations}

\begin{grammar}
\nonterminaldef{constant-declaration} \\
	\terminal{def} \nonterminal{constant-bindings} \\

\nonterminaldef{constant-bindings} \\
	\nonterminal{constant-binding} \optional{\terminal{,}} \\
	\nonterminal{constant-bindings} \terminal{,} \nonterminal{constant-binding} \\

\nonterminaldef{constant-binding} \\
	\nonterminal{identifier} \terminal{:} \nonterminal{type} \terminal{=} \nonterminal{simple-expression} \\
\end{grammar}

\specsubsubitem
In a \nonterminal{constant-declaration}, the \nonterminal{identifier}s in the
\nonterminal{constant-binding} shall be available to the translation
environment. No storage shall be allocated for them in the execution
environment, and they shall not be addressable. References to them shall be
equivalent to references to the \nonterminal{simple-expression} associated with
them, with a cast to \nonterminal{type} inserted.

\specsubsubitem
A \nonterminal{constant-binding}'s \nonterminal{simple-expression} shall be
limited to the \secref{Translation compatible expression subset}, and shall be
evaluated in the translation environment. The type of the value of the
\nonterminal{simple-expression} shall assignable to \nonterminal{type}.

\specsubsection{Type declarations}

\begin{grammar}
\nonterminaldef{type-declaration} \\
	\terminal{type} \nonterminal{type-bindings} \\

\nonterminaldef{type-bindings} \\
	\nonterminal{identifier} \terminal{=} \nonterminal{type} \optional{\terminal{,}} \\
	\nonterminal{identifier} \terminal{=} \nonterminal{type} \terminal{,} \nonterminal{type-bindings} \\
\end{grammar}

\specsubsubitem
In a \nonterminal{type-declaration}, the \nonterminal{identifier}s shall declare
type aliases. In a \nonterminal{type-binding}, the underlying type for the
\nonterminal{identifier} shall be the \nonterminal{type}.

\specsubsection{Function declarations}

\begin{grammar}
\nonterminaldef{function-declaration} \\
	\optional{\nonterminal{fndec-attrs}} \terminal{fn}
		\nonterminal{identifier} \nonterminal{prototype} \terminal{=}
		\nonterminal{complex-expression} \\

\nonterminaldef{fndec-attrs} \\
	\nonterminal{fndec-attr} \\
	\nonterminal{fndec-attr} \nonterminal{fndec-attrs} \\

\nonterminaldef{fndec-attr} \\
	\terminal{@fini} \\
	\terminal{@init} \\
	\terminal{@test} \\
	\nonterminal{fntype-attr} \\
	\nonterminal{decl-attr} \\
\end{grammar}

% TODO: @test
% TODO: Prototypes

\specsubsubitem
The \terminal{@fini} form of \nonterminal{fndec-attr} shall cause the
function to be a finalization function. \terminal{@init} shall cause it to be an
initialization function. If multiple \nonterminal{fndec-attr}s of the same type
are specified, the last one shall override all previous ones.

References: \secref{Initialization functions}, \secref{Finalization functions}

\specsubsubitem
If \terminal{@noreturn} is not specified by the prototype, the function's
expression shall terminate.

\specsubsubitem
If \terminal{@init}, \terminal{@fini}, or \terminal{@test} are given, the
result type shall be \terminal{void}.

\specsection{Units}

\begin{grammar}
\nonterminaldef{sub-unit}\\
	\optional{\nonterminal{imports}} \nonterminal{declarations}\\

\nonterminaldef{imports}\\
	\nonterminal{use-statement}\\
	\nonterminal{use-statement} \nonterminal{imports}\\

\nonterminaldef{use-statement}\\
	\terminal{use} \nonterminal{identifier} \terminal{;}\\
	\terminal{use} \nonterminal{name} \terminal{=} \nonterminal{identifier} \terminal{;}\\
	\terminal{use} \nonterminal{identifier} \terminal{::} \terminal{\{} \nonterminal{name-list} \terminal{\}} \terminal{;}\\

\nonterminaldef{name-list}\\
	\nonterminal{name} \optional{\terminal{,}}\\
	\nonterminal{name} \terminal{,} \nonterminal{name-list}\\
\end{grammar}

\specsubitem
A unit, or translation unit, is composed of several source files as described
by \secref{Translation steps}. Each source file is a \nonterminal{sub-unit}.

\specsubitem
An \nonterminal{import} shall declare a dependency between this translation
unit and another module of the namespace specified by the
\nonterminal{use-statement} \nonterminal{identifier}. This shall cause the named
module to be linked into the final program image as described by
\secref{Translation steps}.

\specsubitem
The first form of the \nonterminal{use-statement} shall cause the identifiers
declared by the target module to become visible to this \nonterminal{sub-unit}
in their fully-qualified form.

\specsubitem
The second form of the \nonterminal{use-statement} shall cause the identifiers
declared by the target module to become visible to this \nonterminal{sub-unit}
in a rewritten form, with the fully-qualified namespace of the identifiers
being visible under the namespace described by the \nonterminal{name} given in
this form.

\informative{In the use statement \code{use foo = bar::baz;}, identifiers in
the namespace \code{bar::baz} will be visible under the namespace \code{foo}.
For example, if the fully-qualified identifier \code{bar::baz::bat} exists, this
\nonterminal{sub-unit} may refer to it as \code{foo::bat}.}

\specsubitem
The third form of the \nonterminal{use-statement} shall cause only the
identifiers listed in the \nonterminal{name-list}, qualified in the context
of the target namespace, to become visible in their un-qualified form to this
\nonterminal{sub-unit}.

\informative{If the use statement \code{use bar::baz::\{bat\}} were specified
in the same conditions as the previous example, the fully-qualified identifier
\code{bar::baz::bat} may be referred to by its unqualified name \code{bat} in
the scope of this \nonterminal{sub-unit}.}

\specsubitem
The imports of a \nonterminal{sub-unit} are not visible to other
\nonterminal{sub-unit}s in the same translation unit. However, identifiers
\textit{declared} in this \nonterminal{sub-unit} are visible in those
\nonterminal{sub-unit}s.

