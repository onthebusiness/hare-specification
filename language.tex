\specchapter{Language}

\specsection{Notation}

\informative{A summary of the language syntax is given in \appxref{Language
syntax summary}.}

\specsubitem
The notation used in this specification indicates non-terminals with
\textit{italic type}, terminals with \terminal{bold type}, and optional
symbols use "opt" in subscript. Non-terminals referenced in the text use the
\under{expression} notation. The following example denotes an optional
\nonterminal{expression} enclosed in literal braces:

\begin{grammar}
\terminal{\{} \optional{\nonterminal{expression}} \terminal{\}}
\end{grammar}

\specsubitem
When there are multiple options for a single non-terminal, they will either be
printed on successive lines, or the preceeding authoritative text will use the
key phrase "one of".

\specsubitem
Most nonterminals are tolerant of white-space characters inserted between their
terminals. However, some are not---these will use the key phrase "exactly" in
their grammar description.

\specsubitem
A non-terminal is defined with its name, a colon (`:'), and the options;
indented and shown with one option per line. For example,
\nonterminal{switch-cases} is defined like so:

\begin{grammar}
\nonterminaldef{switch-cases} \\
	\nonterminal{switch-case} \terminal{;} \\
	\nonterminal{switch-case} \terminal{;} \nonterminal{switch-cases} \\
\end{grammar}

\specsubitem
When the \code{U+XXXX} notation is used, where \code{XXXX} denotes any number of
hexadecimal digits, the hexadecimal digits are to be interpreted as the value of
the denoted Unicode codepoint.

\example{\code{U+000A} denotes the Unicode codepoint with the value 10 (line
feed).}

\specsubitem
Additionally, text may appear in the notation without italics or bold font; it
appears in the same style as the authoritative text. Such examples are used to
describe how a particular terminal sequence is matched when enumerating all of
the possibilities is not practical.

\begin{grammar}
\nonterminaldef{rawstring-char} \\
	\norm{Any character other than \terminal{\textasciigrave}} \\
\end{grammar}

\specsection{Lexical analysis}

\begin{grammar}
\nonterminaldef{token}\\
	\nonterminal{comment} \\
	\nonterminal{constant} \\
	\nonterminal{keyword} \\
	\nonterminal{name} \\
	\nonterminal{operator} \\
	\nonterminal{attribute}

\nonterminaldef{operator}\oneof \\
	\terminal{!}
	\terminal{!=}
	\terminal{\%}
	\terminal{\%=}
	\terminal{\&}
	\terminal{\&\&}
	\terminal{\&\&=}
	\terminal{\&=}
	\terminal{(}
	\terminal{)}
	\terminal{*}
	\terminal{*=}
	\terminal{+}
	\terminal{+=}
	\terminal{,}
	\terminal{-}
	\terminal{-=}
	\terminal{.}
	\terminal{..}
	\terminal{...}
	\terminal{/}
	\terminal{/=}
	\terminal{:}
	\terminal{::}
	\terminal{;}
	\terminal{<}
	\terminal{\textless{}\textless{}}
	\terminal{\textless{}\textless{}=}
	\terminal{<=}
	\terminal{=}
	\terminal{==}
	\terminal{=>}
	\terminal{>}
	\terminal{>=}
	\terminal{\textgreater{}\textgreater{}}
	\terminal{\textgreater{}\textgreater{}=}
	\terminal{?}
	\terminal{[}
	\terminal{]}
	\terminal{\textasciicircum}
	\terminal{\textasciicircum=}
	\terminal{\textasciicircum\textasciicircum}
	\terminal{\textasciicircum\textasciicircum=}
	\terminal{\{}
	\terminal{|}
	\terminal{|=}
	\terminal{||}
	\terminal{||=}
	\terminal{\}}
	\terminal{\textasciitilde}

\nonterminaldef{attribute}\oneof \\
	\terminal{@fini}
	\terminal{@init}
	\terminal{@offset}
	\terminal{@packed}
	\terminal{@symbol}
	\terminal{@test}
	\terminal{@threadlocal}

\nonterminaldef{comment}\exactly\\
	\terminal{//} \nonterminal{comment-chars} \code{U+000A}

\nonterminaldef{comment-chars}\exactly\\
	\nonterminal{comment-char} \optional{\nonterminal{comment-chars}}

\nonterminaldef{comment-char}\\
	\norm{Any character other than \code{U+000A}}

\end{grammar}

\specsubitem
A token is the smallest unit of meaning in the Hare grammar. The lexical
analysis phase processes a \hbox{UTF-8} source file to produce a stream of
tokens by matching the terminals with the input text.

\specsubitem
Tokens may be separated by \textit{white-space} characters, which are defined as
the Unicode codepoints \code{U+0009} (horizontal tabulation), \code{U+000A}
(line feed), and \code{U+0020} (space). Any number of white-space characters may
be inserted between tokens, either to disambiguate from subsequent tokens, or
for aesthetic purposes. This white-space is discarded during the lexical
analysis phase.

\informative{Within a single token, white-space is meaningful. For example, the
\nonterminal{string-literal} token is defined by two quotation marks
\terminal{"} enclosing any number of literal characters. The enclosed
characters are considered part of the \nonterminal{string-literal} token and
any white-space therein is not discarded.}

\specsubitem
The lexical analysis process shall repeatedly consume Unicode characters from
the source file input until there are no more characters to consume. White-space
characters shall be discarded. When a non-white-space character is encountered,
it shall mark the beginning of a token: the longest sequence of characters which
constitutes a token shall then be consumed and emitted to the token stream,
unless the token is a \nonterminal{comment}, in which case it shall be
discarded. If no token can be formed, a diagnostic message shall be printed and
the translation phase shall abort.

\specsection{Keywords}

\begin{grammar}
\nonterminaldef{keyword}\oneof \\
\terminal{alloc}
\terminal{assert}
\terminal{as}
\terminal{bool}
\terminal{break}
\terminal{char}
\terminal{const}
\terminal{continue}
\terminal{def}
\terminal{defer}
\terminal{else}
\terminal{enum}
\terminal{export}
\terminal{f32}
\terminal{f64}
\terminal{false}
\terminal{fn}
\terminal{for}
\terminal{free}
\terminal{i16}
\terminal{i32}
\terminal{i64}
\terminal{i8}
\terminal{if}
\terminal{int}
\terminal{is}
\terminal{len}
\terminal{let}
\terminal{match}
\terminal{nullable}
\terminal{null}
\terminal{offset}
\terminal{return}
\terminal{size}
\terminal{static}
\terminal{struct}
\terminal{str}
\terminal{switch}
\terminal{true}
\terminal{u16}
\terminal{u32}
\terminal{u64}
\terminal{u8}
\terminal{uintptr}
\terminal{uint}
\terminal{union}
\terminal{use}
\terminal{void}
\terminal{while}
\end{grammar}

\specsubitem
Keywords (or \textit{reserved words}) are terminals with special meaning. These
names case-sensitive. Keywords are are reserved for elements of the syntax and
shall not appear in user-defined names, in particular in \secref{Identifiers}.

\specsection{Attributes}

\begin{grammar}
\nonterminaldef{attribute}\oneof \\
	\terminal{@fini}
	\terminal{@init}
	\terminal{@offset}
	\terminal{@packed}
	\terminal{@symbol}
	\terminal{@test}
	\terminal{@threadlocal}

\nonterminaldef{invalid-attribute}\exactly \\
	\terminal{@} \nonterminal{name}
\end{grammar}

\specsubitem
Attributes are terminals with special meaning. They are case-sensitive.

\specsubitem
\nonterminal{invalid-attribute} isn't used anywhere in the syntax. If an
\nonterminal{invalid-attribute} is consumed during lexical analysis, the program
is invalid, and the translation phase shall print a diagnostic message and
abort.

\informative{The purpose of \nonterminal{invalid-attribute} is to disallow the
use of a \nonterminal{keyword} or \nonterminal{name} immediately after an
\nonterminal{attribute}, unless the tokens are separated by white-space. Thus,
the following program is invalid:}

\begin{codesample}
@testfn test() void = void;
\end{codesample}

\specsubsection{Identifiers}

\begin{grammar}
\nonterminaldef{identifier}
\nonterminal{name}\\
\nonterminal{name} \terminal{::} \nonterminal{identifier}
\end{grammar}

\specsubsubitem
An \textit{identifier} is a user-defined name which denotes a module, object,
function, type alias, struct or union member, or enumeration member.

\specsubsubitem
An identifier is only meaningful within a specific \textit{scope} of the
program. The scope is defined by the region of the AST in which the identifier
is applicable; it may be the program, a translation unit, a function, or an
\nonterminal{expression-list}. The identifier is considered \textit{visible}
within the region that defines its scope.

\specsubsubitem
A translation unit is assigned a unique \textit{namespace} within the program.
These namespaces may be nested recursively; that is to say that a translation
unit may have a \textit{parent} which is another translation unit. One
translation unit may be assigned to the \textit{root namespace}, which has no
name.

\specsubsubitem
Identifiers declared within a translation unit scope are assigned the namespace
of the translation unit. The double-colon terminal \terminal{::} is used to
denote the namespace of an identifier, ordered from most to least specific.

\specsubsubitem
An identifier is either \textit{fully-qualified} or \textit{unqualified}.
Unqualified identifiers require the context of their enclosing scope to be
interpreted unambiguously. Fully-qualified identifiers are used for
\textit{exported} identifiers, and include the namespace in which they reside.

\example{The fully-qualified identifier \code{sys::start::start\_ha} qualifies
the un-qualified identifier \code{start\_ha} with the start namespace, which is
itself a member of the sys namespace.}

\specsubsubitem
An identifier without the namespace qualification may be fully-qualified
regardless, if it exists in the root namespace.

\specsection{Types}

\begin{grammar}
\nonterminaldef{type} \\
	\nonterminal{scalar-type} \\
	\nonterminal{struct-union-type} \\
	\nonterminal{tagged-union-type} \\
	\nonterminal{slice-array-type} \\
	\nonterminal{enum-type} \\
	\nonterminal{function-type} \\
	\nonterminal{pointer-type} \\
	\nonterminal{alias-type} \\
	\terminal{const} \nonterminal{type} \\
\end{grammar}

\specsubitem
A type defines the storage and semantics of a value. The attributes common to
all types are its \textit{size}, in octets; \textit{alignment}; and its
\textit{constant} or \textit{mutable} nature.

\specsubitem
The \terminal{const} terminal, when used in a type specifier, enables the
constant flag and prohibits write operations on any value of that type. Types
without this attribute are considered mutable by default.

\specsubsection{Scalar types}

\begin{grammar}
\nonterminaldef{scalar-type} \\
	\nonterminal{integer-type} \\
	\nonterminal{floating-type} \\
	\terminal{bool} \\
	\terminal{null} \\
	\terminal{void} \\
\end{grammar}

\specsubsubitem
A scalar type, also called a \textit{built-in} or \textit{primitive} type,
stores one value at a specific, pre-defined precision. Scalar types are the
most basic unit in the Hare type system.

\informative{This is in contrast to \subsecref{Aggregate types}, which may store
multiple values.}

\specsubsection{Integer types}

\specsubsection{Floating point types}

\specsubsection{Other primitive types}

\specsubsection{Aggregate types}

\specsection{Expressions}

\specsubitem
An expression is a procedure which the implementation may perform to obtain a
\textit{result}, and possibly cause side-effects (see
\subsubitemref{Program execution}{1}).

\specsubitem
Expression types are organized into a number of classes and subclasses of
expressions which define the contexts in which each expression type is
applicable.

\specsubitem
All expressions have a defined \textit{result type}. It may be \terminal{void}.

\specsubitem
Some expressions \textit{terminate}. The semantics of terminating expressions
vary between different expression types, and will be detailed as appropriate.
If unspecified, expressions described by this expression are presumed to be
non-terminating.

\specsubsection{Constants}

\begin{grammar}
\nonterminaldef{constant} \\
	\nonterminal{integer-constant} \\
	\nonterminal{floating-constant} \\
	\nonterminal{rune-constant} \\
	\nonterminal{string-constant} \\
	\terminal{true} \\
	\terminal{false} \\
	\terminal{null} \\
	\terminal{void} \\
\end{grammar}

\specsubsubitem
Constants (also known as literals) shall describe a specific value of an
unambiguous type. The result of the expression is the constant value.

\specsubsubitem
The keywords \terminal{true} and \terminal{false} respectively represent
the constants of the \terminal{bool} type.

\specsubsubitem
The representation of \terminal{true} as an \terminal{uint}-equivalent (ref
\subsubitemref{Other primitive types}{2}) shall be one.

\specsubsubitem
The \terminal{null} keyword represents the \terminal{null} value of the
\terminal{null} type.

\specsubsubitem
The \terminal{void} keyword represents the \terminal{void} value of the
\terminal{void} type.

\specsubsection{Floating constants}

\begin{grammar}
\nonterminaldef{floating-constant} \\
	\nonterminal{decimal-digits} \terminal{.} \nonterminal{decimal-digits} \optional{\nonterminal{exponent}} \optional{\nonterminal{floating-suffix}} \\
	\nonterminal{decimal-digits} \optional{\nonterminal{exponent}} \nonterminal{floating-suffix} \\

\nonterminaldef{floating-suffix} \oneof \\
	\terminal{f32}
	\terminal{f64} \\

\nonterminaldef{decimal-digits} \\
	\nonterminal{decimal-digit} \optional{\nonterminal{decimal-digits}} \\

\nonterminaldef{decimal-digit} \oneof \\
	\terminal{0}
	\terminal{1}
	\terminal{2}
	\terminal{3}
	\terminal{4}
	\terminal{5}
	\terminal{6}
	\terminal{7}
	\terminal{8}
	\terminal{9} \\

\nonterminaldef{exponent} \\
	\terminal{e} \nonterminal{decimal-digits} \\
\end{grammar}

% TODO: consider using magic precision numeric constants which assume
% their type from context

Floating constants represent an IEEE 754-compatible floating-point
number in either the binary32 or binary64 format.

\specsubitem
If the \nonterminal{floating-suffix} is not provided, the type shall be
\terminal{f64}. Otherwise, the type shall refer to the type named by the suffix.

\specsubitem
If the \nonterminal{exponent} is provided, the value of the
constant shall be multiplied by 10 to the power of
\nonterminal{decimal-digits}.

\specsubsection{Integer constants}

\begin{grammar}
\nonterminaldef{integer-constant} \\
	\terminal{0x} \nonterminal{hex-digits} \optional{\nonterminal{integer-suffix}} \\
	\terminal{0o} \nonterminal{octal-digits} \optional{\nonterminal{integer-suffix}} \\
	\terminal{0b} \nonterminal{binary-digits} \optional{\nonterminal{integer-suffix}} \\
	\nonterminal{decimal-digits} \optional{\nonterminal{exponent}}
		\optional{\nonterminal{integer-suffix}} \\

\nonterminaldef{hex-digits} \\
	\nonterminal{hex-digit} \optional{\nonterminal{hex-digits}} \\

\nonterminaldef{hex-digit} \oneof \\
	\terminal{0}
	\terminal{1}
	\terminal{2}
	\terminal{3}
	\terminal{4}
	\terminal{5}
	\terminal{6}
	\terminal{7}
	\terminal{8}
	\terminal{9}
	\terminal{A}
	\terminal{B}
	\terminal{C}
	\terminal{D}
	\terminal{E}
	\terminal{F}
	\terminal{a}
	\terminal{b}
	\terminal{c}
	\terminal{d}
	\terminal{e}
	\terminal{f} \\

\nonterminaldef{octal-digits} \\
	\nonterminal{octal-digit} \optional{\nonterminal{octal-digits}} \\

\nonterminaldef{octal-digit} \oneof \\
	\terminal{0}
	\terminal{1}
	\terminal{2}
	\terminal{3}
	\terminal{4}
	\terminal{5}
	\terminal{6}
	\terminal{7} \\

\nonterminaldef{binary-digits} \\
	\nonterminal{binary-digit} \optional{\nonterminal{binary-digits}} \\

\nonterminaldef{binary-digit} \oneof \\
	\terminal{0}
	\terminal{1} \\

\nonterminaldef{integer-suffix} \oneof \\
	\terminal{i}
	\terminal{u}
	\terminal{z}
	\terminal{i8}
	\terminal{i16}
	\terminal{i32}
	\terminal{i64}
	\terminal{u8}
	\terminal{u16}
	\terminal{u32}
	\terminal{u64} \\
\end{grammar}

% TODO: Consider using magic precision numeric constants which assume their
% type from context

Integer constants represent an integer value at a specific precision.

\specsubitem
If the \nonterminal{integer-suffix} is not provided, the type shall be
\terminal{int}.  Otherwise, the type is specified by the suffix. Suffixes
\terminal{i}, \terminal{u},  and \terminal{z} shall respectively refer to the
\terminal{int}, \terminal{uint}, and \terminal{size} types; the remainder shall
refer to the type named by the suffix.

\specsubitem
If the number provided is not within the limits of the precision of the
constant type, a diagnostic message shall be printed and the translation phase
shall fail.

\specsubitem
The prefixes \terminal{0x}, \terminal{0o}, and \terminal{0b} shall respectively
cause the number to be interpreted with a hexadecimal, octal, or binary base.
If no prefix is used, the number shall be interpreted with a decimal base.

\specsubitem
If the \nonterminal{exponent} is provided, the value of the
integer shall be multiplied by 10 to the power of \nonterminal{decimal-digits}.

\specsubsection{Rune constants}
\begin{grammar}
\nonterminaldef{rune-constant} \\
	\terminal{'} \nonterminal{rune} \terminal{'} \\

\nonterminaldef{rune} \\
	\norm{Any character other than \terminal{\textbackslash} or \terminal{'}} \\
	\nonterminal{escape-sequence} \\

\nonterminaldef{escape-sequence} \\
	\nonterminal{named-escape} \\
	\terminal{\textbackslash{}x} \nonterminal{hex-digit} \nonterminal{hex-digit} \\
	\terminal{\textbackslash{}u} \nonterminal{hex-digit} \nonterminal{hex-digit} \nonterminal{hex-digit} \nonterminal{hex-digit} \\

\nonterminaldef{named-escape} \oneof \\
	\terminal{\textbackslash0}
	\terminal{\textbackslash{}a}
	\terminal{\textbackslash{}b}
	\terminal{\textbackslash{}f}
	\terminal{\textbackslash{}n}
	\terminal{\textbackslash{}r}
	\terminal{\textbackslash{}t}
	\terminal{\textbackslash{}v}
	\terminal{\textbackslash\textbackslash}
	\terminal{\textbackslash'}
	\terminal{\textbackslash"} \\
\end{grammar}

% TODO: Describe representation of rune types

\specsubitem
If the \nonterminal{rune-constant} is not an \nonterminal{escape-sequence}, the
value of the rune shall be the Unicode codepoint representing
\nonterminal{rune}.

\specsubitem
A \nonterminal{rune-constant} beginning with \terminal{\textbackslash{}x} or
\terminal{\textbackslash{}u} shall interpet its value as a Unicode codepoint
specified in its hexadecimal representation by \nonterminal{hex-digit}s.

\specsubitem
A \nonterminal{rune-constant} containing a \nonterminal{named-escape} shall have
a value based on the following chart:

\begin{tabular}{r | l | r | l}
Escape sequence & Unicode codepoint & Escape sequence & Unicode codepoint \\
\hline
\terminal{\textbackslash0} & \code{U+0000} &
\terminal{\textbackslash{}a} & \code{U+0007} \\
\terminal{\textbackslash{}b} & \code{U+0008} &
\terminal{\textbackslash{}f} & \code{U+000C} \\
\terminal{\textbackslash{}n} & \code{U+000A} &
\terminal{\textbackslash{}r} & \code{U+000D} \\
\terminal{\textbackslash{}t} & \code{U+0009} &
\terminal{\textbackslash{}v} & \code{U+000B} \\
\terminal{\textbackslash\textbackslash} & \code{U+005C} &
\terminal{\textbackslash'} & \code{U+002C} \\
\terminal{\textbackslash"} & \code{U+0022} \\
\end{tabular}

\specsubsection{String constants}

\begin{grammar}
\nonterminaldef{string-constant} \\
	\terminal{"} \nonterminal{string-chars} \terminal{"} \\

\nonterminaldef{string-chars} \\
	\nonterminal{string-char} \optional{\nonterminal{string-chars}} \\

\nonterminaldef{string-char} \\
	\norm{Any character other than \terminal{\textbackslash} or \terminal{"}} \\
	\nonterminal{escape-sequence} \\
\end{grammar}

\specsubsection{Array literals}

\begin{grammar}
\nonterminaldef{array-literal} \\
	\terminal{[} \nonterminal{array-members} \terminal{]} \\

\nonterminaldef{array-members} \\
	\nonterminal{simple-expression} \optional{\terminal{,}} \\
	\nonterminal{simple-expression} \terminal{...} \optional{\terminal{,}} \\
	\nonterminal{simple-expression} \terminal{,} \nonterminal{array-members} \\
\end{grammar}

Forward references: \subsecref{Simple, complex, and compound expressions}

\specsubsubitem
An \nonterminal{array-literal} expression produces a value of an array type.
The type of each \nonterminal{simple-expression} shall be uniform and shall
determine the member type of the array value, and the length of the array type
shall be defined by the number of members.

\specsubsubitem
The execution environment shall evaluate the \nonterminal{array-members},
ordered such that any side-effects of evaluating the arguments occur in the
order that the members are listed, such that the \textit{N}th member provides
the value for the \textit{N}th array member.

\specsubsubitem
If the \terminal{...} form is used, the result's array type shall be expandable.

\specsubsection{Struct literals}

\begin{grammar}
\nonterminaldef{struct-literal} \\
	\terminal{struct} \terminal{\{} \nonterminal{field-values} \optional{\terminal{,}} \terminal{\}} \\
	\nonterminal{identifier} \terminal{\{} \nonterminal{struct-initializer} \optional{\terminal{,}} \terminal{\}} \\

\nonterminaldef{struct-initializer} \\
	\nonterminal{field-values} \\
	\nonterminal{field-values} \terminal{,} \terminal{...} \\
	\terminal{...} \\

\nonterminaldef{field-values} \\
	\nonterminal{field-value} \\
	\nonterminal{field-values} \terminal{,} \nonterminal{field-value} \\

\nonterminaldef{field-value} \\
	\nonterminal{identifier} \terminal{=} \nonterminal{simple-initializer} \\
	\nonterminal{identifier} \terminal{:} \nonterminal{type} \terminal{=} \nonterminal{initializer} \\
	\nonterminal{struct-literal} \\
\end{grammar}
% TODO: Define initializer and simple-initializer

\specsubsubitem
A \nonterminal{struct-literal} produces a value of a struct type. The first
form is the \textit{plain form}, and the second form is the \textit{named form}.

\specsubsubitem
If the plain form is given, the result type shall be a struct type defined by
the \nonterminal{field-value}s, in order, with their identifiers and types
explicitly specified. The first form of \nonterminal{field-value} shall not be
used in such a struct.

\specsubsubitem
If the named form is given, the \nonterminal{identifier} shall identify a type
alias (see \subsecref{Type aliases}) which refers to a struct type. The result
type shall be this alias type.

\specsubsubitem
If \terminal{...} is not given, \nonterminal{field-values} shall be
\textit{exhaustive}, and include every field of the result type exactly once.
Otherwise, a diagnostic message shall be printed and the translation phase
shall abort.

\specsubsubitem
If \terminal{...} is given, any fields of the result type which are not
included in \nonterminal{field-values} shall be initialized to their default
values.

\specsubsubitem
If the \nonterminal{struct-literal} form of the \nonterminal{field-value} is
given, its fields shall be interpreted as fields of the parent struct.

\informative{The following values are equivalent: \\
\code{struct \{ a: int = 10, b: int = 20 \}} \\
\code{struct \{ a: int = 10, struct \{ b: int = 20 \} \}}}

\specsubsection{Plain expressions}

\begin{grammar}
\nonterminaldef{plain-expression} \\
	\nonterminal{identifier} \\
	\nonterminal{constant} \\
	\nonterminal{array-literal} \\
	\nonterminal{struct-literal} \\

\nonterminaldef{nested-expression} \\
	\nonterminal{plain-expression} \\
	\terminal{(} \nonterminal{complex-expression} \terminal{)} \\
\end{grammar}

Forward references: \subsecref{Simple, complex, and compound expressions}

\specsubsubitem
\nonterminal{plain-expression} is an expression class which represents its
result value "plainly". In the case of constants and literals, the value is
represented by the result of those expressions. In the case of an
\nonterminal{identifier}, the expression produces the value of the identified
object.

\specsubsubitem
\nonterminal{nested-expression} is an expression class provided to allow the
programmer to overcome constraints placed on the valid range of expression
types in certain parts of the grammar, and to overcome undesirable
associativity between operators.

\specsubsection{Calls}

\begin{grammar}
\nonterminaldef{call-expression} \\
	\nonterminal{postfix-expression} \terminal{(} \optional{\nonterminal{argument-list}} \terminal{)} \\

\nonterminaldef{argument-list} \\
	\nonterminal{complex-expression} \optional{\terminal{,}} \\
	\terminal{...} \nonterminal{complex-expression} \optional{\terminal{,}} \\
	\nonterminal{complex-expression} \terminal{,} \nonterminal{argument-list} \\
\end{grammar}

Forward references: \subsecref{Simple, complex, and compound expressions}

\specsubsubitem
A \nonterminal{call-expression} shall invoke a function in the execution
environment and its result shall be a value of the type specified by the
\nonterminal{postfix-expression}'s function result type. This evaluation shall
include any necessary side-effects per \subsubitemref{Program execution}{1}.

\specsubsubitem
The \nonterminal{argument-list} shall be a list of expressions whose types
shall be assignable to the types of the function's parameters, in the order
that they are declared in the function type.

\specsubsubitem
The execution environment shall evaluate the \nonterminal{argument-list},
ordered such that any side-effects of evaluating the arguments occur in the
order that the arguments are listed, to obtain the parameter values required to
invoke the function.

\specsubsubitem
If the invoked function uses Hare-style variadism, the
\nonterminal{argument-list} shall provide zero or more arguments following the
last non-variadic parameter, all of which must be assignable to the type of the
variadic parameter.

\specsubsubitem
If the final argument uses the \terminal{...} form, it must occupy the position of a
variadic parameter and be of a slice or array type. The implementation shall
interpret this value as the list of variadic parameters.

\specsubsubitem
If the invoked function uses C-style variadism, the function may provide zero
or more arguments following the final parameter. These arguments shall be of a
type with a non-zero size, but are otherwise unconstrained.

\specsubsubitem
The specific means by which the invoked function assumes control of the
execution environment, and by which the arguments are provided to it, is
implementation-defined.

\informative{This is generally provided by the target's ABI specification.}

\specsubsection{Assertions}

\begin{grammar}
\nonterminaldef{assertion-expression} \\
	\terminal{assert} \terminal{(} \nonterminal{simple-expression} \terminal{)} \\
	\terminal{assert} \terminal{(} \nonterminal{simple-expression} \terminal{,} \nonterminal{string-constant} \terminal{)} \\
	\terminal{static} \terminal{assert} \terminal{(} \nonterminal{simple-expression} \terminal{)} \\
	\terminal{static} \terminal{assert} \terminal{(} \nonterminal{simple-expression} \terminal{,} \nonterminal{string-constant} \terminal{)} \\
\end{grammar}

Forward references: \subsecref{Simple, complex, and compound expressions}

\specsubsubitem
An \nonterminal{assertion-expression} is used to validate an assumption by the
programmer by \textit{asserting} its truth. The result type of a
\nonterminal{assertion-expression} is \terminal{void}.

\specsubsubitem
\nonterminal{simple-expression} shall be an expression of type
\nonterminal{bool}.

\specsubsubitem
In the first two forms, this expression shall be evaluated
in the execution environment, and if false, a diagnostic message shall be
printed and the execution phase aborted. The programmer may provide the
\nonterminal{string-constant} to be included in the diagnostic message.

\specsubsubitem
In the \terminal{static} form, \nonterminal{simple-expression} shall be limited
to the \subsecref{Translation compatible expression subset}, shall be evaluated
in the translation environment, and is otherwise equivalent to the other forms.

\specsubsection{Measurements}

\begin{grammar}
\nonterminaldef{measurement-expression} \\
	\nonterminal{size-expression} \\
	\nonterminal{length-expression} \\
	\nonterminal{offset-expression} \\

\nonterminaldef{size-expression} \\
	\terminal{size} \terminal{(} \nonterminal{type} \terminal{)} \\

\nonterminaldef{length-expression} \\
	\terminal{len} \terminal{(} \nonterminal{identifier} \terminal{)} \\

\nonterminaldef{offset-expression} \\
	\terminal{offset} \terminal{(} \nonterminal{field-access-expression} \terminal{)} \\
\end{grammar}

Forward references: \subsecref{Field access}

\specsubsubitem
A \nonterminal{measurement-expression} is used to measure objects. The result
type shall be \terminal{size}.

\specsubsubitem
The \terminal{size} expression shall compute the \textit{size} of the specified
\nonterminal{type}.

\specsubsubitem
The \terminal{len} expression shall compute the \textit{length} of a bounded
array, or the \code{length} field of a slice object, referred to by
\nonterminal{identifier}. If an unbounded array object is given, the
translation environment shall print a diagnostic message and abort.

\specsubsubitem
The \terminal{offset} expression shall determine the struct field which would be
accessed by \nonterminal{field-access-expression} and compute its
\textit{offset}.

\specsubsection{Field access}

\specsubsection{Indexing}

\begin{grammar}
\nonterminaldef{indexing-expression} \\
	\nonterminal{postfix-expression} \terminal{[} \nonterminal{simple-expression} \terminal{]} \\
\end{grammar}

\specsubsubitem
An \nonterminal{indexing-expression} shall access a specific value of a slice
or array type. The \nonterminal{postfix-expression} shall have a result type of
slice or array, and the \nonterminal{simple-expression} shall have a result
type which is assignable to \nonterminal{size}.

\specsubsubitem
The result type of an \nonterminal{indexing-expression} is the secondary type
of the slice or array type given by the \nonterminal{postfix-expression} result
type.

\specsubsubitem
The execution environment shall compute the result of
\nonterminal{simple-expression} to obtain $N$ for selecting the \textit{N} per
the algorithm given in \subsecref{Slice and array types}.

\specsubsubitem
The execution environment shall perform a \textit{bounds test} on the value of
$N$ to ensure it falls within the acceptable range for the given slice or array
type. It shall test that $N < Z$, where $Z$ is the length of the bounded array
type, or the \code{length} field of the slice, whichever is appropriate.  For
unbounded array types, the bounds test shall not occur. If the bounds test
fails, a diagnostic message shall be printed and the execution environment
shall abort.

The implementation may perform a bounds test in the translation environment if
is able, print a diagnostic message, and abort the translation environment if
it fails.

\specsubsection{Slicing}

\begin{grammar}
\nonterminaldef{slicing-expression} \\
	\nonterminal{postfix-expression} \terminal{[} \optional{\nonterminal{simple-expression}} \terminal{..} \optional{\nonterminal{simple-expression}} \terminal{]} \\
\end{grammar}

\specsubsubitem
A \nonterminal{slicing-expression} shall have a result type of
\nonterminal{slice}, which is computed a subset of a slice or array object.
\nonterminal{postfix-expression} shall be of a slice or array type, and the
optional \nonterminal{simple-expression}s shall be assignable to
\nonterminal{size}.

\specsubsubitem
The first \nonterminal{simple-expression} shall compute value $L$, and the
second shall compute $H$. If absent, $L = 0$ and $H = \code{length}$, where
\code{length} shall be equal to the length of a bounded array type or the
length of a slice type, represented in either case by the result of
\nonterminal{postfix-expression}. If $H$ is not specified, and
\nonterminal{postfix-expression} is of an unbounded array type, the translation
environment shall abort.

\specsubsubitem
The resulting slice value shall have its \code{data} field set from, in the case
of an array type, the address of the array; or in the case of a slice type, the
\code{data} value of the source object; plus $L \times S$, where $S$ is the
size of the slice or array's secondary type.

\specsubsubitem
The resulting slice value shall have its \code{length} and \code{capacity}
fields set to $H - L$.

\specsubsubitem
The secondary type of the resulting slice type shall be equivalent to the
secondary type of the slice or array type given by
\nonterminal{postfix-expression}. The resulting slice type shall inherit the
\nonterminal{const} attribute from this type.

\specsubsection{Postfix expressions}

\begin{grammar}
\nonterminaldef{postfix-expression} \\
	\nonterminal{nested-expression} \\
	\nonterminal{assertion-expression} \\
	\nonterminal{call-expression} \\
	\nonterminal{field-access-expression} \\
	\nonterminal{indexing-expression} \\
	\nonterminal{measurement-expression} \\
	\nonterminal{slicing-expression} \\
\end{grammar}

\specsubsubitem
\nonterminal{postfix-expression} is an expression class for expressions whose
operators use postfix notation.

\specsubsection{Unary arithmetic}

\begin{grammar}
\nonterminaldef{unary-expression} \\
	\nonterminal{postfix-expression} \\
	\nonterminal{unary-operator} \nonterminal{unary-expression} \\

\nonterminaldef{unary-operator} \oneof \\
	\terminal{+}
	\terminal{-}
	\terminal{\textasciitilde}
	\terminal{!}
	\terminal{*}
	\terminal{\&}
\end{grammar}

% TODO: Consider removing unary +

\specsubsubitem
A unary expression applies a \nonterminal{unary-operator} to a single value.

\specsubsubitem
The \terminal{+} and \terminal{-} operators shall respectively perform unary
positive and unary negation operations. The result type shall be equivalent to
the type of \nonterminal{unary-expression}, which shall be of a signed numeric
type.

\specsubsubitem
The \terminal{\textasciitilde} operator shall perform a binary NOT operation,
inverting each bit of the value. Its result type shall be equivalent to the
type of \nonterminal{unary-expression}, which shall be of an unsigned integer
type.

\specsubsubitem
The \terminal{!} operator shall perform a logical NOT operation. The result
type, and the type of \nonterminal{unary-expression}, shall both be
\terminal{bool}. If the \nonterminal{unary-expression} is \terminal{true}, the
result shall be \terminal{false}, and vice-versa.

\specsubsubitem
The \terminal{*} operator shall dereference a pointer, and return the object it
references.  The type of \nonterminal{unary-expression} shall be a pointer
type, and the result type shall be the pointer's secondary type. The pointer
type shall not be \textit{nullable}.

\specsubsubitem
The \terminal{\&} operator shall take the address of an object. The result type
shall be a pointer whose secondary type is the type of
\nonterminal{unary-expression}.

\informative{The following table is informative.}

\begin{tabular}{r | l }
Operator & Meaning \\
\hline
\terminal{+} & Positive \\
\terminal{-} & Negation \\
\terminal{\textasciitilde} & Binary NOT \\
\terminal{!} & Logical NOT \\
\terminal{*} & Dereference pointer \\
\terminal{\&} & Take address \\
\end{tabular}

\specsubsection{Casts and type assertions}

\specsubsection{Multiplicative arithmetic}

\specsubsection{Additive arithmetic}

\specsubsection{Bit shifting arithmetic}

\specsubsection{Relational arithmetic}

\specsubsection{Equality}

\specsubsection{Bitwise arithmetic}

\specsubsection{Logical arithmetic}

\specsubsection{Scope expressions}

\specsubsection{If expressions}

\specsubsection{For loops}

\specsubsection{While loops}

\specsubsection{Match expressions}

\specsubsection{Switch expressions}

\specsubsection{Branching expressions}

\specsubsection{Assignment}

\specsubsection{Variable binding}

\specsubsection{Expression lists}

\specsubsection{Simple, complex, and compound expressions}

\specsection{Declarations}

\begin{grammar}
\nonterminaldef{declarations} \\
	\optional{\terminal{export}} \nonterminal{declaration} \terminal{;} \optional{\nonterminal{declarations}} \\
	\nonterminal{static-assertion-expression} \terminal{;} \optional{\nonterminal{declarations}} \\

\nonterminaldef{declaration} \\
	\nonterminal{global-declaration} \\
	\nonterminal{constant-declaration} \\
	\nonterminal{type-declaration} \\
	\nonterminal{function-declaration} \\
\end{grammar}

A \nonterminal{declaration} specifies the interpretation and attributes of a set
of \nonterminal{identifier}s.

\specsubitem
The \nonterminal{identifier}s shall be visible anywhere within the current
translation unit. If the \terminal{export} keyword is used, the
\nonterminal{identifier}s shall be part of the unit's exported interface.

\specsubitem
The \terminal{export} keyword shall not be used with a
\nonterminal{function-declaration} which uses the \terminal{@init},
\terminal{@fini}, or \terminal{@test} attributes.

\specsubsection{Global declarations}

\begin{grammar}
\nonterminaldef{global-declaration} \\
	\terminal{let} \nonterminal{global-bindings} \\
	\terminal{const} \nonterminal{global-bindings} \\

\nonterminaldef{global-bindings} \\
	\nonterminal{global-binding} \optional{\terminal{,}} \\
	\nonterminal{global-bindings} \terminal{,} \nonterminal{global-binding} \\

\nonterminaldef{global-binding} \\
	\optional{\nonterminal{decl-attr}} \optional{\terminal{@threadlocal}} \nonterminal{identifier} \terminal{:} \nonterminal{type} \\
	\optional{\nonterminal{decl-attr}} \optional{\terminal{@threadlocal}} \nonterminal{identifier} \terminal{:} \nonterminal{type} \terminal{=} \nonterminal{expression} \\
	\optional{\nonterminal{decl-attr}} \optional{\terminal{@threadlocal}} \nonterminal{identifier} \terminal{=} \nonterminal{expression} \\

\nonterminaldef{decl-attr} \\
	\terminal{@symbol} \terminal{(} \nonterminal{string-constant} \terminal{)} \\
\end{grammar}

\specsubsubitem
In a \nonterminal{global-declaration}, sufficient space shall be reserved for
each \nonterminal{identifier} in the \nonterminal{global-bindings} to store
the type associated with it. That storage shall be initialized to the value of
the \nonterminal{expression} and shall have alignment greater than or equal to
the necessary alignment for the type. In the \terminal{const} form, the types
shall have the constant flag enabled by default.

\specsubsubitem
A \nonterminal{global-binding}'s \nonterminal{expression} shall be limited to
the \secref{Translation compatible expression subset}, and shall be evaluated
in the translation environment. If specified, the type of the value of the
\nonterminal{expression} shall be assignable to \nonterminal{type}. If not
specified, the type of the \nonterminal{global-binding} shall be the result
of the \nonterminal{expression}. Bindings whose type has undefined size shall
not be provided an \nonterminal{expression}.

\specsubsubitem
The first form of \nonterminal{global-binding} is a \textit{prototype}. In this
form, the implementation shall not allocate storage for the global, and the
programmer must arrange for storage to be provided elsewhere, the manner of
which is implementation-defined.

\specsubsubitem
The interpretation of the \terminal{@symbol} form of \nonterminal{decl-attr} is
implementation-defined. \terminal{@symbol} shall not be used alongside
\terminal{@init}, \terminal{@fini}, or \terminal{@test}.

\informative{The purpose of this directive is to allow users to customize the
symbol name emitted for targets like ELF.}

\specsubsubitem
The interpretation of \terminal{@threadlocal} is implementation-defined.

\informative{The purpose of this directive is to store a separate copy of a
global for each thread, similar to thread\_local in C.}

\specsubsection{Constant declarations}

\begin{grammar}
\nonterminaldef{constant-declaration} \\
	\terminal{def} \nonterminal{constant-bindings} \\

\nonterminaldef{constant-bindings} \\
	\nonterminal{constant-binding} \optional{\terminal{,}} \\
	\nonterminal{constant-bindings} \terminal{,} \nonterminal{constant-binding} \\

\nonterminaldef{constant-binding} \\
	\nonterminal{identifier} \terminal{:} \nonterminal{type} \terminal{=} \nonterminal{expression} \\
	\nonterminal{identifier} \terminal{=} \nonterminal{expression} \\
\end{grammar}

\specsubsubitem
In a \nonterminal{constant-declaration}, the \nonterminal{identifier}s in the
\nonterminal{constant-binding} shall be available to the translation
environment. No storage shall be allocated for them in the execution
environment, and they shall not be addressable. References to them shall be
equivalent to references to the \nonterminal{expression} associated with them,
with a cast to \nonterminal{type} inserted.

\specsubsubitem
A \nonterminal{constant-binding}'s \nonterminal{expression} shall be limited to
the \secref{Translation compatible expression subset}, and shall be evaluated
in the translation environment. If the first form of
\nonterminal{constant-binding} is given, the type of the value of the
\nonterminal{expression} shall be assignable to \nonterminal{type}.

\specsubsection{Type declarations}

\begin{grammar}
\nonterminaldef{type-declaration} \\
	\terminal{type} \nonterminal{type-bindings} \\

\nonterminaldef{type-bindings} \\
	\nonterminal{type-binding} \optional{\terminal{,}} \\
	\nonterminal{type-binding} \terminal{,} \nonterminal{type-bindings} \\

\nonterminaldef{type-binding} \\
	\nonterminal{identifier} \terminal{=} \nonterminal{type} \\
	\nonterminal{identifier} \terminal{=} \nonterminal{enum-type} \\

\nonterminaldef{enum-type} \\
	\terminal{enum} \optional{\nonterminal{enum-storage}} \terminal{\{} \nonterminal{enum-values} \terminal{\}} \\

\nonterminaldef{enum-values} \\
	\nonterminal{enum-value} \optional{\terminal{,}} \\
	\nonterminal{enum-value} \terminal{,} \nonterminal{enum-values} \\

\nonterminaldef{enum-value} \\
	\nonterminal{name} \\
	\nonterminal{name} \terminal{=} \nonterminal{expression} \\

\nonterminaldef{enum-storage} \\
	\nonterminal{integer-type} \\
	\terminal{rune} \\

\end{grammar}

\specsubsubitem
In a \nonterminal{type-declaration}, the \nonterminal{identifier}s shall
declare type aliases. In the first form of \nonterminal{type-binding}, the
underlying type for the \nonterminal{identifier} shall be the
\nonterminal{type}. In the second form, the underlying type shall be
\nonterminal{enum-storage}, if specified. Otherwise, the underlying type shall
be \terminal{int}.

\specsubsubitem
In the second form of \nonterminal{type-binding}, the enum values qualified
with the binding's \nonterminal{identifier} shall be made available to the
translation environment. No storage shall be allocated for them in the
execution environment, and they shall not be addressable.

\specsubsubitem
If the \nonterminal{enum-value} does not specify a \nonterminal{expression},
the value assigned to that \nonterminal{name} is equal to the last value
assigned to an \nonterminal{enum-value} of this enum type plus one. If no such
previous value exists, zero is assigned.

\specsubsubitem
An implicitly assigned \nonterminal{enum-value} shall not exceed the precision
of the underlying integer type; if it were to, a diagnostic message shall be
shown instead per \secref{Diagnostics}.

\specsubsubitem
\nonterminal{expression}, if specified, shall be evaluated in the translation
environment and the resulting value shall be assigned to the corresponding
\nonterminal{enum-value}. The \nonterminal{expression} shall be provided the
enum's type's underlying integer type as a type hint. The result type must be
assignable to the enum type's underlying integer type (ref
\subsecref{Assignment}).

\specsubsubitem
A temporary scope shall be allocated while declaring an enum type, and each value
name, in order, shall be made available to that scope.

\informative{This allows the \nonterminal{expression} for each value to refer
to previously declared values.}

\specsubsubitem
Each \nonterminal{enum-value}'s name shall be unique within the set of all
names of \nonterminal{enum-value}s of the \nonterminal{enum-type}. Otherwise, a
diagnostic message shall be printed and the translation phase shall be aborted.

\specsubsubitem
\nonterminal{expression} shall be limited to the
\secref{Translation compatible expression subset}.

\specsubsection{Function declarations}

\begin{grammar}
\nonterminaldef{function-declaration} \\
	\optional{\nonterminal{fndec-attrs}} \terminal{fn} \nonterminal{identifier} \nonterminal{prototype} \\
	\optional{\nonterminal{fndec-attrs}} \terminal{fn} \nonterminal{name} \nonterminal{prototype} \terminal{=} \nonterminal{expression} \\

\nonterminaldef{fndec-attrs} \\
	\nonterminal{fndec-attr} \\
	\nonterminal{fndec-attr} \nonterminal{fndec-attrs} \\

\nonterminaldef{fndec-attr} \\
	\terminal{@fini} \\
	\terminal{@init} \\
	\terminal{@test} \\
	\nonterminal{fntype-attr} \\
	\nonterminal{decl-attr} \\
\end{grammar}

\specsubsubitem
The first form of \nonterminal{function-declaration} is a \textit{prototype},
and shall cause the \nonterminal{identifier} to refer to the function type
described by the \nonterminal{prototype} and the function attributes. The
programmer must arrange for the implementation of this function to be provided
separately, the manner of which is implementation-defined. \terminal{@init},
\terminal{@fini}, and \terminal{@test} shall not be used on prototypes.

\specsubsubitem
The second form of \nonterminal{function-declaration} shall declare a function
and its implementation. The result type of the expression shall be assignable
to the prototype's result type. The function shall be available in the unit
scope by its \nonterminal{name}, and available to other units by forming a
fully-qualified identifier from the unit namespace and the \nonterminal{name}.

\specsubsubitem
In the second form of \nonterminal{function-declaration}, each
\nonterminal{parameter} in the \nonterminal{prototype} which uses the
\nonterminal{name} form shall be available within the \nonterminal{expression}
by its \nonterminal{name}. Those which do not use the \nonterminal{name} form
shall not be made available.

\specsubsubitem
The \terminal{@fini} form of \nonterminal{fndec-attr} shall cause the
function to be a \textit{finalization function}. \terminal{@init} shall cause it
to be an \textit{initialization function}. \terminal{@test} shall cause it to be
a \textit{test function}. If multiple \nonterminal{fndec-attr}s of the same type
are specified, the last one shall override all previous ones.

\specsubsubitem
Functions declared with \terminal{@test}, \terminal{@init}, or \terminal{@fini}
shall accept no parameters, shall return void, shall not be declared with
\terminal{@noreturn}, shall not appear in \nonterminal{object-selector}s, need
not have unique names, and shall not be inserted into the unit's scope.

\specsection{Units}

\begin{grammar}
\nonterminaldef{sub-unit}\\
	\optional{\nonterminal{imports}} \optional{\nonterminal{declarations}}\\

\nonterminaldef{imports}\\
	\nonterminal{use-statement}\\
	\nonterminal{use-statement} \nonterminal{imports}\\

\nonterminaldef{use-statement}\\
	\terminal{use} \optional{\nonterminal{import-alias}} \nonterminal{identifier} \terminal{;}\\
	\terminal{use} \optional{\nonterminal{import-alias}} \nonterminal{identifier} \terminal{::} \terminal{\{} \nonterminal{member-list} \terminal{\}} \terminal{;}\\
	\terminal{use} \nonterminal{identifier} \terminal{::} \terminal{*} \terminal{;}\\

\nonterminaldef{import-alias}\\
	\nonterminal{name} \terminal{=}\\

\nonterminaldef{member-list}\\
	\nonterminal{member} \optional{\terminal{,}}\\
	\nonterminal{member} \terminal{,} \nonterminal{member-list}\\

\nonterminaldef{member}\\
	\nonterminal{name}\\
	\nonterminal{name} \terminal{=} \nonterminal{name}\\
\end{grammar}

\specsubitem
A unit, or translation unit, is composed of several source files as described
by \secref{Translation steps}. Each source file is a \nonterminal{sub-unit}.
A specific sub-unit may have no declarations, but the unit shall contain at
least one declaration among its sub-units.

\specsubitem
An \nonterminal{import} shall declare a dependency between this translation
unit and another module of the namespace specified by the
\nonterminal{use-statement} \nonterminal{identifier}. This shall cause the named
module to be linked into the final program image as described by
\secref{Translation steps}.

\specsubitem
The first form of the \nonterminal{use-statement} shall cause the identifiers
exported by the target module to become visible to this \nonterminal{sub-unit}
in their fully-qualified form. Additionally, if the imported module has more
than one namespace, identifiers of the form "x::y" shall be made available,
where x is the most-specific namespace, and y is each of the exported members
of the target module.

\specsubitem
The second form of the \nonterminal{use-statement} shall cause only the members
listed in the \nonterminal{member-list}, qualified in the context of the target
namespace to become visible in their un-qualified form to this
\nonterminal{sub-unit}.

\specsubitem
A \nonterminal{member} in the first form shall become visible with the name
given by its definition in the target namespace. A \nonterminal{member} in the
second form shall become visible with the name given by the first
\nonterminal{name} in this form.

\informative{In the use statement \code{use bar::baz::\{bat\}}, identifier
\code{bar::baz::bat} may be referred to by its unqualified name \code{bat} in
the scope of this \nonterminal{sub-unit}.}

\specsubitem
The third form of the \nonterminal{use-statement} shall cause all identifiers
qualified in the context of the target namespace to become visible in their
un-qualified form to this \nonterminal{sub-unit}.

\specsubitem
The forms with \terminal{import-alias} shall be equivalent to the forms without
it except that the identifiers shall become visible under the namespace
described by the \nonterminal{name} given in this form.

\informative{In the use statement \code{use foo = bar::baz;}, identifiers in
the namespace \code{bar::baz} will be visible under the namespace \code{foo}.
For example, if the fully-qualified identifier \code{bar::baz::bat} exists, this
\nonterminal{sub-unit} may refer to it as \code{foo::bat}.}

\specsubitem
The translation unit shall establish a scope into which all
unit-local declarations are inserted. Each sub-unit shall establish another
scope as the parent scope of the unit scope, and in this sub-unit
scope, each of the imports used by that sub-unit shall be made available.

\informative{
In other words, declarations made in a sub-unit are visible to other members of
that unit, but imports in a sub-unit are not visible to other sub-units.}


