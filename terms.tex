\specchapter{Terms and definitions}
% Keep me alphabetized

\newcommand{\specterm}[1]{\specitem \term{#1}:}

\specterm{abort} a process in which the \secref{Execution environment}
immediately proceeds to the program teardown step
\subitemref{Execution environment}{3}.

\specterm{alignment} a specific multiple of an octet-aligned storage address at
which some data is required to be stored.

\example{An object with an alignment of 8 may be stored at addresses 8, 16, 32,
and so on; but not at address 4.}

\specterm{character} a single Unicode code-point encoded in the UTF-8 format.

\specterm{implementation-defined} a detail which is not specified by this
document, but which the implementation is required to define.

\specterm{padding} unused octets added to the storage of some data in order to
meet a required alignment. The value of these octets is undefined.

\specterm{size} the number of octets required to represent some data, including
padding.

% TODO: So far this only makes sense in a couple of situations and we may want
% to get rid of it:
\specterm{undefined} a detail for which no definition is provided, neither by
this specification nor by the implementation. Programs which rely on these
details are non-conforming.
