\specsection{Expressions}

\specsubitem
An expression is a procedure which the implementation may perform to obtain a
\textit{result}, and possibly cause side-effects (see
\subsubitemref{Program execution}{1}).

\specsubitem
Expression types are organized into a number of classes and subclasses of
expressions which define the contexts in which each expression type is
applicable.

\specsubitem
All expressions have a defined \textit{result type}. It may be \terminal{void}.

\specsubitem
Some expressions \textit{terminate}. The semantics of terminating expressions
vary between different expression types, and will be detailed as appropriate.
If unspecified, expressions described by this expression are presumed to be
non-terminating.

\specsubsection{Constants}

\begin{grammar}
\nonterminaldef{constant} \\
	\nonterminal{integer-constant} \\
	\nonterminal{floating-constant} \\
	\nonterminal{rune-constant} \\
	\nonterminal{string-constant} \\
	\terminal{true} \\
	\terminal{false} \\
	\terminal{null} \\
	\terminal{void} \\
\end{grammar}

\specsubitem
Constants (also known as literals) shall describe a specific value of an
unambiguous type. The result of the expression is the constant value.

\specsubitem
The keywords \terminal{true} and \terminal{false} respectively represent
the constants of the \terminal{bool} type.

\specsubitem
The representation of \terminal{true} as an \terminal{uint}-equivalent (ref
\subsubitemref{Other primitive types}{2}) shall be one.

\specsubitem
The \terminal{null} keyword represents the \terminal{null} value of the
\terminal{null} type.

\specsubitem
The \terminal{void} keyword represents the \terminal{void} value of the
\terminal{void} type.

\specsubsection{Integer constants}

\begin{grammar}
\nonterminaldef{integer-constant} \\
	\terminal{0x} \nonterminal{hex-digits} \optional{\nonterminal{integer-suffix}} \\
	\terminal{0o} \nonterminal{octal-digits} \optional{\nonterminal{integer-suffix}} \\
	\terminal{0b} \nonterminal{binary-digits} \optional{\nonterminal{integer-suffix}} \\
	\nonterminal{decimal-digits} \optional{\nonterminal{integer-suffix}} \\

\nonterminaldef{hex-digits} \\
	\nonterminal{hex-digit} \optional{\nonterminal{hex-digits}} \\

\nonterminaldef{hex-digit} \oneof \\
	\terminal{0}
	\terminal{1}
	\terminal{2}
	\terminal{3}
	\terminal{4}
	\terminal{5}
	\terminal{6}
	\terminal{7}
	\terminal{8}
	\terminal{9}
	\terminal{A}
	\terminal{B}
	\terminal{C}
	\terminal{D}
	\terminal{E}
	\terminal{F}
	\terminal{a}
	\terminal{b}
	\terminal{c}
	\terminal{d}
	\terminal{e}
	\terminal{f} \\

\nonterminaldef{decimal-digits} \\
	\nonterminal{decimal-digit} \optional{\nonterminal{decimal-digits}} \\

\nonterminaldef{decimal-digit} \oneof \\
	\terminal{0}
	\terminal{1}
	\terminal{2}
	\terminal{3}
	\terminal{4}
	\terminal{5}
	\terminal{6}
	\terminal{7}
	\terminal{8}
	\terminal{9} \\

\nonterminaldef{octal-digits} \\
	\nonterminal{octal-digit} \optional{\nonterminal{octal-digits}} \\

\nonterminaldef{octal-digit} \oneof \\
	\terminal{0}
	\terminal{1}
	\terminal{2}
	\terminal{3}
	\terminal{4}
	\terminal{5}
	\terminal{6}
	\terminal{7} \\

\nonterminaldef{binary-digits} \\
	\nonterminal{binary-digit} \optional{\nonterminal{binary-digits}} \\

\nonterminaldef{binary-digit} \oneof \\
	\terminal{0}
	\terminal{1} \\

\nonterminaldef{integer-suffix} \oneof \\
	\terminal{i}
	\terminal{u}
	\terminal{z}
	\terminal{i8}
	\terminal{i16}
	\terminal{i32}
	\terminal{i64}
	\terminal{u8}
	\terminal{u16}
	\terminal{u32}
	\terminal{u64} \\
\end{grammar}

% TODO: Scientific notation
% TODO: Consider using magic precision numeric constants which assume their
% type from context

Integer constants represent an integer value at a specific precision.

\specsubitem
If the \nonterminal{integer-suffix} is not provided, the type shall be
\terminal{int}.  Otherwise, the type is specified by the suffix. Suffixes
\terminal{i}, \terminal{u},  and \terminal{z} shall respectively refer to the
\terminal{int}, \terminal{uint}, and \terminal{size} types; the remainder shall
refer to the type named by the suffix.

\specsubitem
If the number provided is not within the limits of the precision of the
constant type, a diagnostic message shall be printed and the translation phase
shall fail.

\specsubitem
The prefixes \terminal{0x}, \terminal{0o}, and \terminal{0b} shall respectively
cause the number to be interpreted with a hexadecimal, octal, or binary base.
If no prefix is used, the number shall be interpreted with a decimal base.

\specsubsection{Rune constants}
\begin{grammar}
\nonterminaldef{rune-constant} \\
	\terminal{'} \nonterminal{rune} \terminal{'} \\

\nonterminaldef{rune} \\
	\norm{Any character other than \terminal{\textbackslash} or \terminal{'}} \\
	\nonterminal{escape-sequence} \\

\nonterminaldef{escape-sequence} \\
	\nonterminal{named-escape} \\
	\terminal{\textbackslash{}x} \nonterminal{hex-digit} \nonterminal{hex-digit} \\
	\terminal{\textbackslash{}u} \nonterminal{hex-digit} \nonterminal{hex-digit} \nonterminal{hex-digit} \nonterminal{hex-digit} \\

\nonterminaldef{named-escape} \oneof \\
	\terminal{\textbackslash0}
	\terminal{\textbackslash{}a}
	\terminal{\textbackslash{}b}
	\terminal{\textbackslash{}f}
	\terminal{\textbackslash{}n}
	\terminal{\textbackslash{}r}
	\terminal{\textbackslash{}t}
	\terminal{\textbackslash{}v}
	\terminal{\textbackslash\textbackslash}
	\terminal{\textbackslash'}
	\terminal{\textbackslash"} \\
\end{grammar}

% TODO: Describe representation of rune types

\specsubitem
If the \nonterminal{rune-constant} is not an \nonterminal{escape-sequence}, the
value of the rune shall be the Unicode codepoint representing
\nonterminal{rune}.

\specsubitem
A \nonterminal{rune-constant} beginning with \terminal{\textbackslash{}x} or
\terminal{\textbackslash{}u} shall interpet its value as a Unicode codepoint
specified in its hexadecimal representation by \nonterminal{hex-digit}s.

\specsubitem
A \nonterminal{rune-constant} containing a \nonterminal{named-escape} shall have
a value based on the following chart:

\begin{tabular}{r | l | r | l}
Escape sequence & Unicode codepoint & Escape sequence & Unicode codepoint \\
\hline
\terminal{\textbackslash0} & \code{U+0000} &
\terminal{\textbackslash{}a} & \code{U+0007} \\
\terminal{\textbackslash{}b} & \code{U+0008} &
\terminal{\textbackslash{}f} & \code{U+000C} \\
\terminal{\textbackslash{}n} & \code{U+000A} &
\terminal{\textbackslash{}r} & \code{U+000D} \\
\terminal{\textbackslash{}t} & \code{U+0009} &
\terminal{\textbackslash{}v} & \code{U+000B} \\
\terminal{\textbackslash\textbackslash} & \code{U+005C} &
\terminal{\textbackslash'} & \code{U+002C} \\
\terminal{\textbackslash"} & \code{U+0022} \\
\end{tabular}

\specsubsection{Slice literals}

\specsubsection{Struct literals}

\specsubsection{Immediate expressions}

\specsubsection{Postfix expressions}

\specsubsection{Calls}

\specsubsection{Struct field access}

\specsubsection{Slice member access}

\specsubsection{Built-in functions}

\specsubsection{Unary arithmetic}

\specsubsection{Casts and type assertions}

\specsubsection{Multiplicative arithmetic}

\specsubsection{Additive arithmetic}

\specsubsection{Bit shifting arithmetic}

\specsubsection{Relational arithmetic}

\specsubsection{Equality}

\specsubsection{Bitwise arithmetic}

\specsubsection{Logical arithmetic}

\specsubsection{Scope expressions}

\specsubsection{Assignment}

\specsubsection{Variable binding}

\specsubsection{Expression lists}

\specsubsection{Compound expressions}

\specsubsection{Branching expressions}

\specsubsection{If expressions}

\specsubsection{For loops}

\specsubsection{While loops}

\specsubsection{Match expressions}

\specsubsection{Switch expressions}
