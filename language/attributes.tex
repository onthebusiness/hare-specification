\specsection{Attributes}

\begin{grammar}
\nonterminaldef{attribute}\oneof \\
	\terminal{@fini}
	\terminal{@init}
	\terminal{@offset}
	\terminal{@packed}
	\terminal{@symbol}
	\terminal{@test}
	\terminal{@threadlocal}

\nonterminaldef{invalid-attribute}\exactly \\
	\terminal{@} \nonterminal{name}
\end{grammar}

\specsubitem
Attributes are terminals with special meaning. They are case-sensitive.

\specsubitem
\nonterminal{invalid-attribute} isn't used anywhere in the syntax. If an
\nonterminal{invalid-attribute} is consumed during lexical analysis, the program
is invalid, and the translation phase shall print a diagnostic message and
abort.

\informative{The purpose of \nonterminal{invalid-attribute} is to disallow the
use of a \nonterminal{keyword} or \nonterminal{name} immediately after an
\nonterminal{attribute}, unless the tokens are separated by white-space. Thus,
the following program is invalid:}

\begin{codesample}
@testfn test() void = void;
\end{codesample}
