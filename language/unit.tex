\specsection{Declarations}

\begin{grammar}
\nonterminaldef{declarations} \\
	\optional{\terminal{export}} \nonterminal{declaration} \terminal{;} \optional{\nonterminal{declarations}} \\
	\nonterminal{static-assertion-expression} \terminal{;} \optional{\nonterminal{declarations}} \\

\nonterminaldef{declaration} \\
	\nonterminal{global-declaration} \\
	\nonterminal{constant-declaration} \\
	\nonterminal{type-declaration} \\
	\nonterminal{function-declaration} \\
\end{grammar}

\specsubitem
A \nonterminal{declaration} specifies the interpretation and attributes of a set
of \nonterminal{identifier}s.

\specsubitem
If the \terminal{export} keyword is used, the declaration is considered to be
\textit{exported}, allowing other modules to access it. An exported declaration
may be visible within the scopes of multiple translation units.

\specsubitem
The \terminal{export} keyword shall not be used with a
\nonterminal{function-declaration} which uses the \terminal{@init},
\terminal{@fini}, or \terminal{@test} attributes.

\specsubsection{Global declarations}

\begin{grammar}
\nonterminaldef{global-declaration} \\
	\terminal{let} \nonterminal{global-bindings} \\
	\terminal{const} \nonterminal{global-bindings} \\

\nonterminaldef{global-bindings} \\
	\nonterminal{global-binding} \optional{\terminal{,}} \\
	\nonterminal{global-binding} \terminal{,} \nonterminal{global-bindings} \\

\nonterminaldef{global-binding} \\
	\optional{\nonterminal{decl-attr}} \optional{\terminal{@threadlocal}} \nonterminal{identifier} \terminal{:} \nonterminal{type} \\
	\optional{\nonterminal{decl-attr}} \optional{\terminal{@threadlocal}} \nonterminal{identifier} \terminal{:} \nonterminal{type} \terminal{=} \nonterminal{expression} \\
	\optional{\nonterminal{decl-attr}} \optional{\terminal{@threadlocal}} \nonterminal{identifier} \terminal{=} \nonterminal{expression} \\

\nonterminaldef{decl-attr} \\
	\terminal{@symbol} \terminal{(} \nonterminal{string-literal} \terminal{)} \\
\end{grammar}

\specsubsubitem
In a \nonterminal{global-declaration}, sufficient space shall be reserved for
each \nonterminal{global-binding} to store the type associated with it. That
storage shall be initialized to the value of the \nonterminal{expression} and
shall have alignment greater than or equal to the necessary alignment for the
type. In the \terminal{const} form, the types shall have the constant flag
enabled by default.

\specsubsubitem
The \nonterminal{identifier} of each \nonterminal{global-binding} shall be
inserted into the current translation unit's scope, for use by any other
declaration in the translation unit. If the \nonterminal{identifier} already
refers to an object visible in the translation unit's scope, a diagnostic
message shall be printed and the translation phase shall be aborted.

\specsubsubitem
A \nonterminal{global-binding}'s \nonterminal{expression} shall be limited to
the \secref{Translation compatible expression subset}, and shall be evaluated
in the translation environment. If specified, the type of the value of the
\nonterminal{expression} shall be assignable to \nonterminal{type}. If not
specified, the type of the \nonterminal{global-binding} shall be the result
of the \nonterminal{expression}. Bindings whose type has undefined size shall
not be provided an \nonterminal{expression}. If the type of a binding is a
flexible type, it shall first be lowered to its default type.

\specsubsubitem
The first form of \nonterminal{global-binding} is a \textit{prototype}. In this
form, the implementation shall not allocate storage for the global, and the
programmer must arrange for storage to be provided elsewhere, the manner of
which is implementation-defined.

\specsubsubitem
The interpretation of the \terminal{@symbol} form of \nonterminal{decl-attr} is
implementation-defined. \terminal{@symbol} shall not be used alongside
\terminal{@init}, \terminal{@fini}, or \terminal{@test}.

\informative{The purpose of this directive is to allow users to customize the
symbol name emitted for targets like ELF.}

\specsubsubitem
The interpretation of \terminal{@threadlocal} is implementation-defined.

\informative{The purpose of this directive is to store a separate copy of a
global for each thread, similar to thread\_local in C.}

\specsubsection{Constant declarations}

\begin{grammar}
\nonterminaldef{constant-declaration} \\
	\terminal{def} \nonterminal{constant-bindings} \\

\nonterminaldef{constant-bindings} \\
	\nonterminal{constant-binding} \optional{\terminal{,}} \\
	\nonterminal{constant-binding} \terminal{,} \nonterminal{constant-bindings} \\

\nonterminaldef{constant-binding} \\
	\nonterminal{identifier} \terminal{:} \nonterminal{type} \terminal{=} \nonterminal{expression} \\
	\nonterminal{identifier} \terminal{=} \nonterminal{expression} \\
\end{grammar}

\specsubsubitem
In a \nonterminal{constant-declaration}, the \nonterminal{identifier}s in the
\nonterminal{constant-binding} shall be available to the translation
environment. No storage shall be allocated for them in the execution
environment. References to them shall be equivalent to references to the
\nonterminal{expression} associated with them, with a cast to \nonterminal{type}
inserted if the first form is used.

\informative{When the second form is used, \nonterminal{expression} may have
a flexible result type.}

\specsubsubitem
The \nonterminal{identifier} of each \nonterminal{constant-binding} shall be
inserted into the current translation unit's scope, for use by any other
declaration in the translation unit. If the \nonterminal{identifier} already
refers to an object visible in the translation unit's scope, a diagnostic
message shall be printed and the translation phase shall be aborted.

\specsubsubitem
A \nonterminal{constant-binding}'s \nonterminal{expression} shall be limited to
the \secref{Translation compatible expression subset}, and shall be evaluated
in the translation environment. If the first form of
\nonterminal{constant-binding} is given, the type of the value of the
\nonterminal{expression} shall be assignable to \nonterminal{type}.

\specsubsection{Type declarations}

\begin{grammar}
\nonterminaldef{type-declaration} \\
	\terminal{type} \nonterminal{type-bindings} \\

\nonterminaldef{type-bindings} \\
	\nonterminal{type-binding} \optional{\terminal{,}} \\
	\nonterminal{type-binding} \terminal{,} \nonterminal{type-bindings} \\

\nonterminaldef{type-binding} \\
	\nonterminal{identifier} \terminal{=} \nonterminal{type} \\
	\nonterminal{identifier} \terminal{=} \nonterminal{enum-type} \\

\nonterminaldef{enum-type} \\
	\terminal{enum} \optional{\nonterminal{enum-storage}} \terminal{\{} \nonterminal{enum-values} \terminal{\}} \\

\nonterminaldef{enum-values} \\
	\nonterminal{enum-value} \optional{\terminal{,}} \\
	\nonterminal{enum-value} \terminal{,} \nonterminal{enum-values} \\

\nonterminaldef{enum-value} \\
	\nonterminal{name} \\
	\nonterminal{name} \terminal{=} \nonterminal{expression} \\

\nonterminaldef{enum-storage} \\
	\nonterminal{integer-type} \\
	\terminal{rune} \\

\end{grammar}

\specsubsubitem
In a \nonterminal{type-declaration}, the \nonterminal{identifier}s shall
declare type aliases. In the first form of \nonterminal{type-binding}, the
underlying type for the \nonterminal{identifier} shall be the
\nonterminal{type}. In the second form, the underlying type shall be
\nonterminal{enum-storage}, if specified. Otherwise, the underlying type shall
be \terminal{int}.

\specsubsubitem
The \nonterminal{identifier} of each \nonterminal{type-binding} shall be
inserted into the current translation unit's scope, for use by any other
declaration in the translation unit. If the \nonterminal{identifier} already
refers to an object visible in the translation unit's scope, a diagnostic
message shall be printed and the translation phase shall be aborted.

\specsubsubitem
In the second form of \nonterminal{type-binding}, each \nonterminal{enum-value}
shall be made available to the translation unit's scope, via an identifier
comprised of the components of the type alias' identifier followed by the
\nonterminal{enum-value}'s \nonterminal{name}. No storage shall be allocated for
them in the execution environment. References to them shall be equivalent to
references to the \nonterminal{enum-value}'s assigned value.

\specsubsubitem
Each \nonterminal{enum-value} is assigned the value of its
\nonterminal{expression}, if present. Otherwise, the value assigned to it is
equal to the value of the nearest lexically preceding \nonterminal{enum-value}
of this enum type plus one. If no such previous value exists, zero is assigned.

\specsubsubitem
If an implicitly assigned \nonterminal{enum-value} would not be representable in
the underlying integer type, a diagnostic message shall be shown as per
\secref{Diagnostics}.

\specsubsubitem
\nonterminal{expression}, if specified, shall be evaluated in the translation
environment and the resulting value shall be assigned to the corresponding
\nonterminal{enum-value}. The \nonterminal{expression} shall be provided the
enum's type's underlying integer type as a type hint. The result type must be
assignable to the enum type's underlying integer type (ref
\subsecref{Assignment}).

\specsubsubitem
The implementation shall establish a new scope for the
\nonterminal{enum-values}, and each \nonterminal{enum-value} name shall be made
available in the scope.

\informative{This allows the \nonterminal{expression} for each value to refer
to other values within the enum.}

\specsubsubitem
Each \nonterminal{enum-value}'s name shall be unique within the set of all
names of \nonterminal{enum-value}s of the \nonterminal{enum-type}. Otherwise, a
diagnostic message shall be printed and the translation phase shall be aborted.

\specsubsubitem
\nonterminal{expression} shall be limited to the
\secref{Translation compatible expression subset}.

\specsubsection{Function declarations}

\begin{grammar}
\nonterminaldef{function-declaration} \\
	\optional{\nonterminal{fndec-attrs}} \terminal{fn} \nonterminal{identifier} \nonterminal{prototype} \\
	\optional{\nonterminal{fndec-attrs}} \terminal{fn} \nonterminal{identifier} \nonterminal{prototype} \terminal{=} \nonterminal{expression} \\

\nonterminaldef{fndec-attrs} \\
	\nonterminal{fndec-attr} \\
	\nonterminal{fndec-attr} \nonterminal{fndec-attrs} \\

\nonterminaldef{fndec-attr} \\
	\terminal{@fini} \\
	\terminal{@init} \\
	\terminal{@test} \\
	\nonterminal{decl-attr} \\
\end{grammar}

\specsubsubitem
The \nonterminal{identifier} of each \nonterminal{function-declaration} which
does not use \terminal{@init}, \terminal{@fini}, or \terminal{@test} shall be
inserted into the current translation unit's scope, for use by any other
declaration in the translation unit. If the \nonterminal{identifier} already
refers to an object visible in the translation unit's scope, a diagnostic
message shall be printed and the translation phase shall be aborted.

\specsubsubitem
The first form of \nonterminal{function-declaration} is a \textit{prototype},
and shall cause the \nonterminal{identifier} to refer to the function type
described by the \nonterminal{prototype} and the function attributes. The
programmer must arrange for the implementation of this function to be provided
separately, the manner of which is implementation-defined. \terminal{@init},
\terminal{@fini}, and \terminal{@test} shall not be used on prototypes.

\specsubsubitem
The second form of \nonterminal{function-declaration} shall declare a function
and its implementation. The result type of the expression shall be assignable
to the \nonterminal{prototype}'s result type.

\specsubsubitem
In the second form of \nonterminal{function-declaration}, the implementation
shall establish a new scope for the \nonterminal{expression}. For each of the
\nonterminal{prototype}'s \nonterminal{parameters}, in the order that they
appear in the program source, the implementation shall resolve the parameter's
type (within the newly established scope), and then, if the \nonterminal{name}
form is used, insert the parameter's \nonterminal{name} into the scope. The
\nonterminal{prototype}'s result type shall be resolved outside of the newly
established scope. The \nonterminal{expression} shall be translated after all
named parameters have been made visible.

\specsubsubitem
The \terminal{@fini} form of \nonterminal{fndec-attr} shall cause the
function to be a \textit{finalization function}. \terminal{@init} shall cause it
to be an \textit{initialization function}. \terminal{@test} shall cause it to be
a \textit{test function}. If multiple \nonterminal{fndec-attr}s of the same type
are specified, the last one shall override all previous ones.

\specsubsubitem
Functions declared with \terminal{@test}, \terminal{@init}, or \terminal{@fini}
shall accept no parameters, shall return void, and need not have unique names.

\specsection{Units}

\begin{grammar}
\nonterminaldef{sub-unit}\\
	\optional{\nonterminal{imports}} \optional{\nonterminal{declarations}}\\

\nonterminaldef{imports}\\
	\nonterminal{use-directive} \optional{\nonterminal{imports}}\\

\nonterminaldef{use-directive}\\
	\terminal{use} \nonterminal{identifier} \terminal{;}\\
	\terminal{use} \nonterminal{name} \terminal{=} \nonterminal{identifier} \terminal{;}\\
	\terminal{use} \nonterminal{identifier} \terminal{::} \terminal{\{} \nonterminal{member-list} \terminal{\}} \terminal{;}\\
	\terminal{use} \nonterminal{identifier} \terminal{::} \terminal{*} \terminal{;}\\

\nonterminaldef{member-list}\\
	\nonterminal{name} \optional{\terminal{,}}\\
	\nonterminal{name} \terminal{,} \nonterminal{member-list}\\
\end{grammar}

\specsubitem
A unit, or translation unit, is composed of several source files as described
by \secref{Translation steps}. Each source file is a \nonterminal{sub-unit}.
A specific sub-unit may have no declarations, but the unit shall contain at
least one \nonterminal{declaration} (excluding static assertions) among its
sub-units.

\specsubitem
Each translation unit shall establish a scope into which all of the unit's
declarations are inserted. Each sub-unit shall also establish a scope, into
which any \nonterminal{imports} shall make declarations from other modules
available.

\informative{
In other words, declarations made in a sub-unit are visible to other members of
that unit, but imports in a sub-unit are not visible to other sub-units.}

\specsubitem
A \nonterminal{use-directive} shall declare a dependency between this module and
another module (the \textit{target module}) whose namespace is specified by the
\nonterminal{use-directive}'s identifier. This shall cause the named module to
be included in the final program, as described by \secref{Translation steps}.
All dependencies of the target module also become dependencies of this module.

\specsubitem
If a \nonterminal{name} or \nonterminal{identifier} which would be inserted into
this sub-unit's scope by a \nonterminal{use-directive} already refers to an
object visible in the scope, a diagnostic message shall be printed and the
translation phase shall be aborted.

\specsubitem
If a \nonterminal{use-directive} would cause a module to depend on itself, a
diagnostic message shall be printed and the translation phase shall be aborted.

\specsubitem
The first form of \nonterminal{use-directive} shall cause all declarations
exported by the target module to be made available to this
\nonterminal{sub-unit}'s scope, via an identifier comprised of the last
component of the namespace identifier followed by the components of the
declaration's identifier.

\example{In the use directive \code{use bar::baz;}, identifiers in the module
whose namespace is \code{bar::baz} will be made visible as identifiers whose
first component is \code{baz}. For example, if \code{bar::baz} exports a
declaration named \code{bat}, it is made visible to this \nonterminal{sub-unit}
as \code{baz::bat}.}

\specsubitem
The second form of \nonterminal{use-directive} shall cause all declarations
exported by the target module to be made available to this
\nonterminal{sub-unit}'s scope, via an identifier comprised of the given
\nonterminal{name} followed by the components of the declaration's identifier.

\example{In the use directive \code{use foo = bar::baz;}, identifiers in the
module whose namespace is \code{bar::baz} will be made visible as identifiers
whose first component is \code{foo}. For example, if \code{bar::baz} exports a
declaration named \code{bat}, it is made visible to this \nonterminal{sub-unit}
as \code{foo::bat}.}

\specsubitem
The third form of \nonterminal{use-directive} shall cause only the declarations
named in the \nonterminal{member-list}, each of which shall name a declaration
exported by the target module, to be inserted into this \nonterminal{sub-unit}'s
scope (with the same name they were initially declared with). For each enum type
alias inserted into the scope, all of the enum's members shall also be inserted,
each with the same name that they have in the unit scope of the target module.

\example{If \code{bar::baz} exports a declaration named \code{bat}, the use
directive \code{use bar::baz::\{bat\}}, will cause the name
\code{bat} in this \nonterminal{sub-unit}'s scope to refer to said declaration.}

\specsubitem
The fourth form of \nonterminal{use-directive} shall cause all declarations
exported by the target module to be made available to this
\nonterminal{sub-unit}'s scope, with the same names they were initially declared
with.
