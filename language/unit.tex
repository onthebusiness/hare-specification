\specsection{Declarations}

\begin{grammar}
\nonterminaldef{declarations} \\
	\optional{\terminal{export}} \nonterminal{declaration} \terminal{;} \\
	\optional{\terminal{export}} \nonterminal{declaration} \terminal{;} \nonterminal{declarations} \\

\nonterminaldef{declaration} \\
	\nonterminal{global-declaration} \\
	\nonterminal{constant-declaration} \\
	\nonterminal{type-declaration} \\
	\nonterminal{function-declaration} \\
\end{grammar}

A \nonterminal{declaration} specifies the interpretation and attributes of a set
of \nonterminal{identifier}s.

\specsubitem
The \nonterminal{identifier}s shall be visible anywhere within the current
translation unit. If the \terminal{export} keyword is used, the
\nonterminal{identifier}s shall be part of the unit's exported interface.

\specsubitem
The \terminal{export} keyword shall not be used with a
\nonterminal{function-declaration} which uses the \terminal{@init},
\terminal{@fini}, or \terminal{@test} attributes.

\specsubsection{Global declarations}

\begin{grammar}
\nonterminaldef{global-declaration} \\
	\terminal{let} \nonterminal{global-bindings} \\
	\terminal{const} \nonterminal{global-bindings} \\

\nonterminaldef{global-bindings} \\
	\nonterminal{global-binding} \optional{\terminal{,}} \\
	\nonterminal{global-bindings} \terminal{,} \nonterminal{global-binding} \\

\nonterminaldef{global-binding} \\
	\optional{\nonterminal{decl-attr}} \nonterminal{identifier} \terminal{:} \nonterminal{type} \terminal{=} \nonterminal{simple-expression} \\

\nonterminaldef{decl-attr} \\
	\terminal{@symbol} \terminal{(} \nonterminal{string-constant} \terminal{)}
\end{grammar}

\specsubsubitem
In a \nonterminal{global-declaration}, sufficient space shall be reserved for
each \nonterminal{identifier} in the \nonterminal{global-bindings} to store
the \nonterminal{type} associated with it. That storage shall be initialized to
the value of the \nonterminal{simple-expression} and shall have alignment
greater than or equal to the necessary alignment for the \nonterminal{type}. In
the \terminal{const} form, the \nonterminal{type}s shall have the constant flag
enabled by default.

\specsubsubitem
A \nonterminal{global-binding}'s \nonterminal{simple-expression} shall be
limited to the \secref{Translation compatible expression subset}, and shall be
evaluated in the translation environment. The type of the value of the
\nonterminal{simple-expression} shall assignable to \nonterminal{type}.

\specsubsubitem
The interpretation of the \terminal{@symbol} form of \nonterminal{decl-attr} is
implementation-defined.

\specsubsection{Constant declarations}

\begin{grammar}
\nonterminaldef{constant-declaration} \\
	\terminal{def} \nonterminal{constant-bindings} \\

\nonterminaldef{constant-bindings} \\
	\nonterminal{constant-binding} \optional{\terminal{,}} \\
	\nonterminal{constant-bindings} \terminal{,} \nonterminal{constant-binding} \\

\nonterminaldef{constant-binding} \\
	\nonterminal{identifier} \terminal{:} \nonterminal{type} \terminal{=} \nonterminal{simple-expression} \\
\end{grammar}

\specsubsubitem
In a \nonterminal{constant-declaration}, the \nonterminal{identifier}s in the
\nonterminal{constant-binding} shall be available to the translation
environment. No storage shall be allocated for them in the execution
environment, and they shall not be addressable. References to them shall be
equivalent to references to the \nonterminal{simple-expression} associated with
them, with a cast to \nonterminal{type} inserted.

\specsubsubitem
A \nonterminal{constant-binding}'s \nonterminal{simple-expression} shall be
limited to the \secref{Translation compatible expression subset}, and shall be
evaluated in the translation environment. The type of the value of the
\nonterminal{simple-expression} shall assignable to \nonterminal{type}.

\specsubsection{Type declarations}

\begin{grammar}
\nonterminaldef{type-declaration} \\
	\terminal{type} \nonterminal{type-bindings} \\

\nonterminaldef{type-bindings} \\
	\nonterminal{identifier} \terminal{=} \nonterminal{type} \optional{\terminal{,}} \\
	\nonterminal{identifier} \terminal{=} \nonterminal{type} \terminal{,} \nonterminal{type-bindings} \\
\end{grammar}

\specsubsubitem
In a \nonterminal{type-declaration}, the \nonterminal{identifier}s shall declare
type aliases. In a \nonterminal{type-binding}, the underlying type for the
\nonterminal{identifier} shall be the \nonterminal{type}.

\specsubsection{Function declarations}

\begin{grammar}
\nonterminaldef{function-declaration} \\
	\optional{\nonterminal{fndec-attrs}} \terminal{fn}
		\nonterminal{identifier} \nonterminal{prototype} \\
	\optional{\nonterminal{fndec-attrs}} \terminal{fn}
		\nonterminal{name} \nonterminal{prototype} \terminal{=}
		\nonterminal{complex-expression} \\

\nonterminaldef{fndec-attrs} \\
	\nonterminal{fndec-attr} \\
	\nonterminal{fndec-attr} \nonterminal{fndec-attrs} \\

\nonterminaldef{fndec-attr} \\
	\terminal{@fini} \\
	\terminal{@init} \\
	\terminal{@test} \\
	\nonterminal{fntype-attr} \\
	\nonterminal{decl-attr} \\
\end{grammar}

\specsubsubitem
The first form of \nonterminal{function-declaration} is a \textit{prototype},
and shall cause the \nonterminal{identifier} to refer to the function type
described by the \nonterminal{protototype} and the function attributes. The
implementation of this function shall be provided separately, or the
translation phase shall fail.

\specsubsubitem
The second form of \nonterminal{function-declaration} shall declare a function
and its implementation. The result type of the expression shall be assignable
to the prototype's result type. The function shall be available in the unit
scope by its \nonterminal{name}, and available to other units by forming a
fully-qualified identifier from the unit namespace and the \nonterminal{name}.
In this form, the \nonterminal{name} of each parameter in the
\nonterminal{prototype} is mandatory.

\specsubsubitem
The \terminal{@fini} form of \nonterminal{fndec-attr} shall cause the
function to be a finalization function. \terminal{@init} shall cause it to be an
initialization function. If multiple \nonterminal{fndec-attr}s of the same type
are specified, the last one shall override all previous ones.

References: \secref{Initialization functions}, \secref{Finalization functions}

\specsubsubitem
If \terminal{@init}, \terminal{@fini}, \terminal{@test}, or
\terminal{@noreturn} are given, the result type shall be \terminal{void}.

\specsubsubitem
The \terminal{@test} attribute indicates that a function is used for testing.
The name of the function need not be unique within its namespace, and it shall
not be inserted into the unit's scope. \terminal{@test} functions shall not be
exported. Other semantics of \terminal{@test} functions are
implementation-defined.

\specsection{Units}

\begin{grammar}
\nonterminaldef{sub-unit}\\
	\optional{\nonterminal{imports}} \nonterminal{declarations}\\

\nonterminaldef{imports}\\
	\nonterminal{use-statement}\\
	\nonterminal{use-statement} \nonterminal{imports}\\

\nonterminaldef{use-statement}\\
	\terminal{use} \nonterminal{identifier} \terminal{;}\\
	\terminal{use} \nonterminal{name} \terminal{=} \nonterminal{identifier} \terminal{;}\\
	\terminal{use} \nonterminal{identifier} \terminal{::} \terminal{\{} \nonterminal{name-list} \terminal{\}} \terminal{;}\\

\nonterminaldef{name-list}\\
	\nonterminal{name} \optional{\terminal{,}}\\
	\nonterminal{name} \terminal{,} \nonterminal{name-list}\\
\end{grammar}

\specsubitem
A unit, or translation unit, is composed of several source files as described
by \secref{Translation steps}. Each source file is a \nonterminal{sub-unit}.

\specsubitem
An \nonterminal{import} shall declare a dependency between this translation
unit and another module of the namespace specified by the
\nonterminal{use-statement} \nonterminal{identifier}. This shall cause the named
module to be linked into the final program image as described by
\secref{Translation steps}.

\specsubitem
The first form of the \nonterminal{use-statement} shall cause the identifiers
exported by the target module to become visible to this \nonterminal{sub-unit}
in their fully-qualified form. Additionally, if the imported module has more
than one namespace, identifiers of the form "x::y" shall be made available,
where x is the most-specific namespace, and y is each of the exported members
of the target module.

\specsubitem
The second form of the \nonterminal{use-statement} shall cause the identifiers
declared by the target module to become visible to this \nonterminal{sub-unit}
in a rewritten form, with the fully-qualified namespace of the identifiers
being visible under the namespace described by the \nonterminal{name} given in
this form.

\informative{In the use statement \code{use foo = bar::baz;}, identifiers in
the namespace \code{bar::baz} will be visible under the namespace \code{foo}.
For example, if the fully-qualified identifier \code{bar::baz::bat} exists, this
\nonterminal{sub-unit} may refer to it as \code{foo::bat}.}

\specsubitem
The third form of the \nonterminal{use-statement} shall cause only the
identifiers listed in the \nonterminal{name-list}, qualified in the context
of the target namespace, to become visible in their un-qualified form to this
\nonterminal{sub-unit}.

\informative{If the use statement \code{use bar::baz::\{bat\}} were specified
in the same conditions as the previous example, the fully-qualified identifier
\code{bar::baz::bat} may be referred to by its unqualified name \code{bat} in
the scope of this \nonterminal{sub-unit}.}

\specsubitem
The translation unit shall establish a top-level scope into which all
unit-local declarations are inserted. Each sub-unit shall establish another
scope whose parent scope is the top-level unit scope, and in this sub-unit
scope, each of the imports used by that sub-unit shall be made available.

\informative{
In other words, declarations made in a sub-unit are visible to other members of
that unit, but imports in a sub-unit are not visible to other sub-units.}
