\specsection{Identifiers}

\begin{grammar}
\nonterminaldef{identifier}\\
\nonterminal{name}\\
\nonterminal{name} \terminal{::} \nonterminal{identifier}\\

\nonterminaldef{name}\\
\nonterminal{nondigit}\\
\nonterminal{name} \nonterminal{alnum}\\

\nonterminaldef{nondigit}\oneof \\
\terminal{a} \terminal{b} \terminal{c} \terminal{d} \terminal{e} \terminal{f}
\terminal{g} \terminal{h} \terminal{i} \terminal{j} \terminal{k} \terminal{l}
\terminal{m} \\
\terminal{n} \terminal{o} \terminal{p} \terminal{q} \terminal{r}
\terminal{s} \terminal{t} \terminal{u} \terminal{v} \terminal{w} \terminal{x}
\terminal{y} \terminal{z} \\
\terminal{A} \terminal{B} \terminal{C} \terminal{D} \terminal{E} \terminal{F}
\terminal{G} \terminal{H} \terminal{I} \terminal{J} \terminal{K} \terminal{L}
\terminal{M} \\
\terminal{N} \terminal{O} \terminal{P} \terminal{Q} \terminal{R}
\terminal{S} \terminal{T} \terminal{U} \terminal{V} \terminal{W} \terminal{X}
\terminal{Y} \terminal{Z} \\
\terminal{\_}\\

\nonterminaldef{digit}\oneof \\
\terminal{0} \terminal{1} \terminal{2} \terminal{3} \terminal{4} \terminal{5}
\terminal{6} \terminal{7} \terminal{8} \terminal{9}\\

\nonterminaldef{alnum}\\
\nonterminal{digit}
\nonterminal{nondigit}
\end{grammar}

% TODO: The role of the namespace part could be described better

\specsubitem
An \textit{identifier} is a user-defined name which denotes a module, object,
function, type alias, struct or union member, or enumeration member.

\specsubitem
An identifier is only meaningful within a specific \textit{scope} of the
program. The scope is defined by the region of the AST in which the identifier
is applicable; it may be the program, a translation unit, a sub-unit, a
function, or an \nonterminal{expression-list}. The identifier is considered
\textit{visible} within the region that defines its scope.

\specsubitem
A translation unit is assigned a unique \textit{namespace} within the program.
These namespaces may be nested recursively; that is to say that a translation
unit may have a \textit{parent} which is another translation unit. One
translation unit may be assigned to the \textit{root namespace}, which has no
name.

\specsubitem
Identifiers declared within a translation unit scope are assigned the namespace
of the translation unit. The double-colon terminal \terminal{::} is used to
denote the namespace of an identifier, ordered from most to least specific.

\specsubitem
An identifier is either \textit{fully-qualified} or \textit{unqualified}.
Unqualified identifiers require the context of their enclosing scope to be
interpreted unambiguously. Fully-qualified identifiers are used for
\textit{exported} identifiers, and include the namespace in which they reside.

\example{The fully-qualified identifier \code{sys::start::start\_ha} qualifies
the un-qualified identifier \code{start\_ha} with the start namespace, which is
itself a member of the sys namespace.}

\specsubitem
An identifier without the namespace qualification may be fully-qualified
regardless, if it exists in the root namespace.
