\specsection{Keywords}

\begin{grammar}
\nonterminaldef{keyword}\oneof \\
\terminal{abort}
\terminal{alloc}
\terminal{append}
\terminal{as}
\terminal{assert}
\terminal{bool}
\terminal{break}
\terminal{case}
\terminal{char}
\terminal{const}
\terminal{continue}
\terminal{def}
\terminal{defer}
\terminal{delete}
\terminal{else}
\terminal{enum}
\terminal{export}
\terminal{f32}
\terminal{f64}
\terminal{false}
\terminal{fn}
\terminal{for}
\terminal{free}
\terminal{i16}
\terminal{i32}
\terminal{i64}
\terminal{i8}
\terminal{if}
\terminal{int}
\terminal{is}
\terminal{len}
\terminal{let}
\terminal{match}
\terminal{null}
\terminal{nullable}
\terminal{offset}
\terminal{return}
\terminal{rune}
\terminal{size}
\terminal{static}
\terminal{str}
\terminal{struct}
\terminal{switch}
\terminal{true}
\terminal{type}
\terminal{u16}
\terminal{u32}
\terminal{u64}
\terminal{u8}
\terminal{uint}
\terminal{uintptr}
\terminal{union}
\terminal{use}
\terminal{void}
\terminal{\_}
\end{grammar}

\specsubitem
Keywords (or \textit{reserved words}) are terminals with special meaning. These
names are case-sensitive. Keywords are reserved for elements of the syntax
and shall not appear as user-defined names, in particular in
\secref{Identifiers}.
