\specsection{Lexical analysis}

\begin{grammar}
\nonterminaldef{token}\\
	\nonterminal{comment} \\
	\nonterminal{constant} \\
	\nonterminal{keyword} \\
	\nonterminal{name} \\
	\nonterminal{operator} \\
	\nonterminal{attribute}

\nonterminaldef{operator}\oneof \\
	\terminal{!}
	\terminal{!=}
	\terminal{\%}
	\terminal{\%=}
	\terminal{\&}
	\terminal{\&\&}
	\terminal{\&\&=}
	\terminal{\&=}
	\terminal{(}
	\terminal{)}
	\terminal{*}
	\terminal{*=}
	\terminal{+}
	\terminal{+=}
	\terminal{,}
	\terminal{-}
	\terminal{-=}
	\terminal{.}
	\terminal{..}
	\terminal{...}
	\terminal{/}
	\terminal{/=}
	\terminal{:}
	\terminal{::}
	\terminal{;}
	\terminal{<}
	\terminal{\textless{}\textless{}}
	\terminal{\textless{}\textless{}=}
	\terminal{<=}
	\terminal{=}
	\terminal{==}
	\terminal{=>}
	\terminal{>}
	\terminal{>=}
	\terminal{\textgreater{}\textgreater{}}
	\terminal{\textgreater{}\textgreater{}=}
	\terminal{?}
	\terminal{[}
	\terminal{]}
	\terminal{\textasciicircum}
	\terminal{\textasciicircum=}
	\terminal{\textasciicircum\textasciicircum}
	\terminal{\textasciicircum\textasciicircum=}
	\terminal{\{}
	\terminal{|}
	\terminal{|=}
	\terminal{||}
	\terminal{||=}
	\terminal{\}}
	\terminal{\textasciitilde}

\nonterminaldef{attribute}\oneof \\
	\terminal{@fini}
	\terminal{@init}
	\terminal{@offset}
	\terminal{@packed}
	\terminal{@symbol}
	\terminal{@test}
	\terminal{@threadlocal}

\nonterminaldef{comment}\exactly\\
	\terminal{//} \nonterminal{comment-chars}

\nonterminaldef{comment-chars}\exactly\\
	\nonterminal{comment-char} \optional{\nonterminal{comment-chars}}

\nonterminaldef{comment-char}\\
	\norm{Any character other than \code{U+000A}}

\end{grammar}

\specsubitem
A token is the smallest unit of meaning in the Hare grammar. The lexical
analysis phase processes a \hbox{UTF-8} source file to produce a stream of
tokens by matching the terminals with the input text.

\specsubitem
Tokens may be separated by \textit{white-space} characters, which are defined as
the Unicode codepoints \code{U+0009} (horizontal tabulation), \code{U+000A}
(line feed), and \code{U+0020} (space). Any number of white-space characters may
be inserted between tokens, either to disambiguate from subsequent tokens, or
for aesthetic purposes. This white-space is discarded during the lexical
analysis phase.

\informative{Within a single token, white-space is meaningful. For example, the
\nonterminal{string-literal} token is defined by two quotation marks
\terminal{"} enclosing any number of literal characters. The enclosed
characters are considered part of the \nonterminal{string-literal} token and
any white-space therein is not discarded.}

\specsubitem
The lexical analysis process shall repeatedly consume Unicode characters from
the source file input until there are no more characters to consume. White-space
characters shall be discarded. When a non-white-space character is encountered,
it shall mark the beginning of a token: the longest sequence of characters which
constitutes a token shall then be consumed and emitted to the token stream,
unless the token is a \nonterminal{comment}, in which case it shall be
discarded. If no token can be formed, a diagnostic message shall be printed and
the translation phase shall abort.
