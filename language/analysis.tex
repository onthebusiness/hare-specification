\specsection{Lexical analysis}

\begin{grammar}
% TODO: There are other token non-terminals not represented here yet
\nonterminaldef{token}\\
	\nonterminal{keyword} \\
	\nonterminal{name}
\end{grammar}

\specsubitem
A token is the smallest unit of meaning in the Hare grammar. The lexical
analysis phase processes a UTF-8 source file to produce a stream of tokens by
matching the terminals with the input text.

\specsubitem
Tokens may be separated by \textit{white-space} characters, which are defined as
the Unicode code-points \code{U+0009} (horizontal tabulation), \code{U+000A}
(line feed), and \code{U+0020} (space). Any number of whitespace characters may
be inserted between tokens, either to disambiguate from subsequent tokens, or
for aesthetic purposes. This whitespace is discarded during the lexical
analysis phase.

% TODO: Define the U+XXXX syntax

\informative{Within a single token, white-space is meaningful. For example, the
\nonterminal{string-literal} token is defined by two quotation marks
\terminal{"} enclosing any number of literal characters. The enclosed
characters are considered part of the \nonterminal{string-literal} token and
any whitespace therein is not discarded.}

\specsubitem
The lexical analysis process consumes Unicode characters from the source file
input until it is exhausted, performing the following steps in order. At each
step, it shall consume and discard white-space characters until a
non-white-space characters is found, then consume the longest sequence of
characters which constitutes a token and emit it to the token stream.

\specsubitem
The terminal sequence \terminal{//} is used to mark a comment. When the lexical
analyzer encounters this terminal sequence, it shall discard it and all
subsequent characters until a line feed \code{U+000A} is encountered, then
resume normal processing.
