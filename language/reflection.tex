\specsection{Reflection}

The Hare implementation provides runtime access to information about Hare types
via the \terminal{type} meta-type. Per \subsecref{Casts and type assertions},
the \terminal{type} type may be cast to a pointer. The intended use of this
behavior is to cast it to a "\code{*types::typeinfo}" type to examine the inner
structure of that type. This specification pre-defines the expected structure
of these values.

\specsubitem
The implementation shall cause the pointer represented by a \terminal{type}
value to refer to a \code{types::typeinfo} object allocated according to the
following structure, presuming it were defined in the \code{types} namespace.

\begin{codesample}
export def SIZE_UNDEFINED: size = -1: size;

export type typeinfo = struct {
	id: uint,
	sz: size,
	al: size,
	flags: flags,
	repr: repr,
};

export type flags = enum uint {
	NONE = 0,
	CONST = 1 << 0,
	ERROR = 1 << 1,
};

export type repr = (alias | array | builtin
	| enumerated | func | pointer | slice_repr
	| struct_union | tagged | tuple);

export type alias = struct {
	ident: []str,
	secondary: type,
};

export type array = struct {
	length: size,
	members: type,
};

export type builtin = enum uint {
	BOOL, CHAR, ENUM, F32, F64, I16, I32, I64, I8, INT, NULL, RUNE, SIZE,
	STR, U16, U32, U64, U8, UINT, UINTPTR, VOID, TYPE,
};

export type enumerated = struct {
	storage: builtin,
	values: [](str, union { u: u64, i: i64 }),
};

export type variadism = enum {
	NONE,
	C,
	HARE,
};

export type func_flags = enum uint {
	NORETURN = 1 << 0,
};

export type func = struct {
	result: type,
	variadism: variadism,
	flags: func_flags,
	params: []type,
};

export type pointer_flags = enum uint {
	NONE = 0,
	NULLABLE = 1 << 0,
};

export type pointer = struct {
	secondary: type,
	flags: pointer_flags,
};

export type slice_repr = type;

export type struct_kind = enum {
	STRUCT,
	UNION,
};

export type struct_union = struct {
	kind: struct_kind,
	fields: []struct_field,
};

export type struct_field = struct {
	name: str,
	offs: size,
	type_: type,
};

export type tagged = []type;

export type tuple = []tuple_value;

export type tuple_value = struct {
	offs: size,
	type_: type,
};
\end{codesample}

\specsubsection{types::typeinfo}

\specsubsubitem
The \code{types::typeinfo} structure shall define the following members:

\begin{itemize}
\item \textbf{id}: A unique identifier for this type
\item \textbf{sz}: The size, in octets, of this type
\item \textbf{al}: The alignment, in octets, of this type
\item \textbf{flags}: A bitfield describing the type flags
\item \textbf{repr}: A tagged union describing the type representation
\end{itemize}

See also: \subitemref{Types}{2}

\specsubsection{types::flags}

\specsubsubitem
The typeinfo flags field shall be populated according to the flags of the type,
such that the ERROR bit is set if the type has the error flag, and the CONST
bit is set if the type has the constant flag.

\specsubsection{types::repr}

\specsubsubitem
The typeinfo repr field shall be a tagged union which selects a structure
describing the representation of the type. The implementation shall note that
the type ID selected is chosen based on the fully qualified type identifier for
the structure aliases, such that a struct or union type selects the
\code{types::struct\_union} value, rather than a \code{struct\_union} alias in the
current unit's namespace.

\specsubsection{types::alias}

\specsubsubitem
A type value describing a type alias shall select the \code{types::alias} tag
of the \code{types::typeinfo.repr} field. The \code{types::alias} type shall be
a struct defining the following fields:

\begin{itemize}
\item \textbf{ident}: a slice of \terminal{str} containing each part of the
fully qualified identifier for this type alias, in order from most to least
significant part
\item \textbf{type}: a type value describing the type the alias refers to
\end{itemize}

\specsubsection{types::array}

\specsubsubitem
A type value describing an array type shall select the \code{types::array} tag
of the \code{types::typeinfo.repr} field. The \code{types::array} type shall be
a struct defining the following fields:

\begin{itemize}
\item \textbf{length}: the length of the array (\informative{i.e. the result
value of a \nonterminal{length-expression} measuring an array of this type}),
or -1 (\code{SIZE\_UNDEFINED}) if the length is undefined
\item \textbf{members}: a type value describing the secondary type of the array
\end{itemize}

\specsubsection{types::builtin}

\specsubsubitem
If a type value describes any of the following types, it shall shall select the
\code{types::builtin} tag of the \code{types::typeinfo.repr} field:

\begin{grammar}
\nonterminaldef{builtin types}\oneof \\
	\terminal{bool}
	\terminal{char}
	\terminal{f32}
	\terminal{f64}
	\terminal{i16}
	\terminal{i32}
	\terminal{i64}
	\terminal{i8}
	\terminal{int}
	\terminal{null}
	\terminal{rune}
	\terminal{size}
	\terminal{str}
	\terminal{u16}
	\terminal{u32}
	\terminal{u64}
	\terminal{u8}
	\terminal{uint}
	\terminal{uintptr}
	\terminal{void}
	\terminal{type}
\end{grammar}

\specsubsubitem
The \code{types::builtin} type shall be an enum type which defines one value
for each builtin type, where the name of each value is the terminal of that
type with each letter converted to uppercase, and the assigned value ascends
from zero in the order they appear in the list above.

\specsubsection{types::enumerated}

\specsubsubitem
A type value describing an enumerated type shall select the
\code{types::enumerated} tag of the \code{types::typeinfo.repr} field. The
\code{types::enumerated} type shall be a struct defining the following fields:

\begin{itemize}
\item \textbf{storage}: the \code{types::builtin} value corresponding to the
enum's underlying \nonterminal{integer-type}
\item \textbf{values}: a slice of tuples of type
\code{(str, union \{ u: u64, i: i64 \})} which represents the names and values of
the enumerated type, in ascending alphabetical order of their names, sorted
with respect to the ordinal values of the codepoints of each name
\end{itemize}

\specsubsection{types::func}

\specsubsubitem
A type value describing a function type shall select the \code{types::func} tag
of the \code{types::typeinfo.repr} field. The \code{types::func} type shall be
a struct defining the following fields:

\begin{itemize}
\item \textbf{result}: a type value describing the result type of the function
\item \textbf{variadism}: a \code{types::variadism} value describing the variadic style of the function
\item \textbf{flags}: a \code{types::func\_flags} bitfield describing the state the \terminal{@noreturn} flag
\item \textbf{params}: a slice of type values describing the types of the parameters of the function, in order
\end{itemize}

\specsubsubitem
The \code{types::variadism} type shall be an enumerated type defining the
values NONE, C, and HARE as zero, one, and two respectively, referring
respectively to non-variadic functions, functions with C-style variadism, and
functions with Hare-style variadism.

\specsubsubitem
The \code{types::func\_flags} shall define the values NONE and NORETURN, set
respectively to zero and one.

\specsubsection{types::pointer}

\specsubsubitem
A type value describing a pointer type shall select the \code{types::pointer}
tag of the \code{types::typeinfo.repr} field. The \code{types::pointer} type
shall be a struct defining the following fields:

\begin{itemize}
\item \textbf{secondary}: a type value describing the pointer's secondary type
\item \textbf{flags}: a \code{types::pointer\_flags} bitfield describing the state of the \terminal{nullable} flag
\end{itemize}

\specsubsubitem
The \code{types::pointer\_flags} shall define the values NONE and NULLABLE, set
respectively to zero and one.

\specsubsection{types::slice}

\specsubsubitem
A type value describing a slice type shall select the \code{types::slice} tag
of the \code{types::typeinfo.repr} field. The \code{types::slice} type
shall an alias for \terminal{type}, and its value shall be a type value
describing the slice's secondary type.

% Fuck LaTeX: this should be \specsubsection{types::struct\_union}, but that
% breaks everything
\specsubsection{struct and union types}

\specsubsubitem
A type value describing a slice or union type shall select the
\code{types::struct\_union} tag of the \code{types::typeinfo.repr} field. The
\code{types::struct\_union} type shall be a struct defining the following
fields:

\begin{itemize}
\item \textbf{kind}: a \code{types::struct\_kind} value disambiguating between \terminal{struct} and \terminal{union} types
\item \textbf{fields}: a slice of \code{types::struct\_field} values for each field, sorted first by the field's offset, then its name, in ascending order with respect to the offset value or ordinal values of each codepoint in the name
\end{itemize}

\specsubsubitem
The \code{types::struct\_kind} type shall be an enumerated type which defines
the values STRUCT and UNION as zero and one respectively.

\specsubsubitem
The \code{types::struct\_field} type shall be a structure which defines the
following fields:

\begin{itemize}
\item \textbf{name}: a str value set to the name of the field, or empty string if the field is unnamed
\item \textbf{offs}: a size value set to the field's offset in octets
\item \textbf{type\_}: a type value describing the field's type
\end{itemize}

\specsubsection{types::tagged}

\specsubsubitem
A type value describing a slice or union type shall select the
\code{types::tagged} tag of the \code{types::typeinfo.repr} field. The
\code{types::tagged} type shall be a slice of type values representing each
candidate type of the tagged union type, in ascending order of each value's
type ID.

\specsubsection{types::tuple}

\specsubsubitem
A type value describing a slice or union type shall select the
\code{types::tuple} tag of the \code{types::typeinfo.repr} field. The
\code{types::tuple} type shall be a slice of \code{types::tuple\_value} values
representing each value of the tuple type, in order.

\specsubsubitem
The \code{types::tuple\_value} type shall be a structure defining the following
fields:

\begin{itemize}
\item \textbf{offs}: a size value set to the offset of this value in octets
\item \textbf{type\_}: a type value describing this value's type
\end{itemize}

\specsubsection{types::reflect}

\specsubsubitem
The implementation shall define the following function in the types namespace:

\begin{codesample}
export fn reflect(in: type) const *typeinfo = in: *typeinfo;
\end{codesample}

\informative{This is the preferred means for programs to convert a
type value into a \code{*types::typeinfo} reference to examine type details.}
