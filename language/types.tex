\specsection{Types}

\begin{grammar}
\nonterminaldef{type} \\
	\nonterminal{scalar-type} \\
	\nonterminal{struct-union-type} \\
	\nonterminal{tagged-union-type} \\
	\nonterminal{slice-array-type} \\
	\nonterminal{enum-type} \\
	\nonterminal{function-type} \\
	\nonterminal{pointer-type} \\
	\nonterminal{alias-type} \\
	\terminal{const} \nonterminal{type} \\
\end{grammar}

\specsubitem
A type defines the storage and semantics of a value. The attributes common to
all types are its \textit{size}, in octets; \textit{alignment}; and its
\textit{constant} or \textit{mutable} nature.

\specsubitem
The \terminal{const} terminal, when used in a type specifier, enables the
constant flag and prohibits write operations on any value of that type. Types
without this attribute are considered mutable by default.

\specsubsection{Scalar types}

\begin{grammar}
\nonterminaldef{scalar-type} \\
	\nonterminal{integer-type} \\
	\nonterminal{floating-type} \\
	\terminal{bool} \\
	\terminal{null} \\
	\terminal{void} \\
\end{grammar}

\specsubsubitem
A scalar type, also called a \textit{built-in} or \textit{primitive} type,
stores one value at a specific, pre-defined precision. Scalar types are the
most basic unit in the Hare type system.

\informative{This is in contrast to \subsecref{Aggregate types}, which may store
multiple values.}

\specsubsection{Integer types}

\specsubsection{Floating point types}

\specsubsection{Other primitive types}

\specsubsection{Aggregate types}
