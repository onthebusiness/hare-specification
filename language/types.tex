\specsection{Types}

\begin{grammar}
\nonterminaldef{type} \\
	\nonterminal{scalar-type} \\
	\nonterminal{struct-union-type} \\
	\nonterminal{tagged-union-type} \\
	\nonterminal{slice-array-type} \\
	\nonterminal{pointer-type} \\
	\nonterminal{function-type} \\
	\nonterminal{alias-type} \\
	\terminal{const} \nonterminal{type} \\

\nonterminaldef{scalar-type} \\
	\nonterminal{integer-type} \\
	\nonterminal{floating-type} \\
	\nonterminal{enum-type} \\
	\terminal{bool} \\
	\terminal{null} \\
	\terminal{void} \\
\end{grammar}

\specsubitem
A type defines the storage and semantics of a value. The attributes common to
all types are its \textit{size}, in octets; \textit{alignment}; and its
\textit{constant} or \textit{mutable} nature.

\specsubitem
Some types have an undefined size. This includes \nonterminal{function-type},
and some cases of \nonterminal{slice-array-type}.

\specsubitem
The \terminal{const} terminal, when used in a type specifier, enables the
constant flag and prohibits write operations on any value of that type. Types
without this attribute are considered mutable by default.

\specsubitem
A scalar type, also called a \textit{built-in} or \textit{primitive} type,
stores one value at a specific, pre-defined precision. Scalar types are the
most basic unit in the Hare type system. Other types are referred to as
\textit{aggregate types}, with the exception of alias types, which may be
either scalar or aggregate.

\specsubsection{Integer types}

\specsubsection{Floating point types}

\specsubsection{Enum types}

\specsubsection{Other primitive types}

\specsubsection{Struct and union types}

\specsubsection{Tagged union types}

\specsubsection{Slice and array types}

\specsubsection{Pointer types}

\specsubsection{Function types}

\specsubsection{Type aliases}
