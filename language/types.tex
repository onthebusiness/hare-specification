\specsection{Types}

\begin{grammar}

\nonterminaldef{type} \\
	\optional{\terminal{const}} \optional{\terminal{!}} \nonterminal{storage-class} \\

\nonterminaldef{storage-class} \\
	\nonterminal{primitive-type} \\
	\nonterminal{pointer-type} \\
	\nonterminal{struct-union-type} \\
	\nonterminal{tuple-type} \\
	\nonterminal{tagged-union-type} \\
	\nonterminal{slice-array-type} \\
	\nonterminal{function-type} \\
	\nonterminal{alias-type} \\
	\nonterminal{unwrapped-alias} \\

\nonterminaldef{primitive-type} \\
	\nonterminal{integer-type} \\
	\nonterminal{floating-type} \\
	\terminal{bool} \\
	\terminal{never} \\
	\terminal{opaque} \\
	\terminal{rune} \\
	\terminal{str} \\
	\terminal{valist} \\
	\terminal{void} \\
\end{grammar}

\specsubitem
A type defines the storage and semantics of a value. The attributes common to
all types are its \textit{size}, in octets; its \textit{alignment}, in octets;
its \textit{constant} or \textit{mutable} nature; its \textit{error flag}, or
lack thereof; and its \textit{default value}. The size, alignment, and default
value of a type may be undefined.

\specsubitem
The implementation shall assign a globally unique ID to every type, in a
deterministic manner, such that several subsequent translation environments,
perhaps with different inputs, will obtain the same unique ID; and such that
distinct types shall have distinct IDs.

This specification details under what circumstances two types are equivalent to
one another, and thus shall have the same ID. For all types, any two types are
distinct if their type class, their constant or mutable nature, their error
flag or lack thereof, or their default values, are distinct. Each type class
may impose additional distinguishing characteristics on their types, which are
specified in their respective sections.

\specsubitem
Some types have an undefined size. This includes \nonterminal{function-type},
and some cases of \nonterminal{slice-array-type}.

\specsubitem
Some types have an undefined default value. If the default value of such a type
would be used, the implementation shall instead print a diagnostic message and
abort the translation phase.

\specsubitem
The \terminal{const} terminal, when used in a type specifier, enables the
constant flag and prohibits write operations on any value of that type. Types
without this attribute are considered mutable by default.

\specsubitem
The \terminal{!} terminal, when used in a type specifier, sets the
error flag for this type.

\specsubitem
The \textit{type class} of a type is defined for primitive types as the terminal
which represents it, for example \terminal{i32}.

\specsubsection{Integer types}

\begin{grammar}
\nonterminaldef{integer-type}\oneof \\
	\terminal{i8}
	\terminal{i16}
	\terminal{i32}
	\terminal{i64}
	\terminal{u8}
	\terminal{u16}
	\terminal{u32}
	\terminal{u64}
	\terminal{int}
	\terminal{uint}
	\terminal{size}
	\terminal{uintptr}
\end{grammar}

\specsubsubitem
Integer types represent an integer value at a specific width. These values are
either \textit{signed} or \textit{unsigned}; which respectively \textbf{are} and
\textbf{are not} able to represent negative integers. Integer types are
considered \textit{numeric types}.

\specsubsubitem
Signed integer types shall be represented in two's complement form, such that
arithmetic on signed operands behaves identically to arithmetic on unsigned
operands. The most significant bit of a signed integer shall determine the sign
of the value: the value shall be positive if the most significant bit is unset,
otherwise it shall be negative. Zero shall be positive.

\specsubsubitem
For integer types whose size is greater than one octet, the order in which its
octets are represented is implementation-defined, provided that the significance
strictly either increases or decreases as the memory address increases.

\specsubsubitem
If an operation on an integer type would cause the result to overflow, it is
truncated towards the least significant bit.

\specsubsubitem
The width in bits of \terminal{i8}, \terminal{i16}, \terminal{i32},
\terminal{i64}, \terminal{u8}, \terminal{u16}, \terminal{u32}, and
\terminal{u64} are specified by the numeric suffix. Of those, types prefixed
with \code{u} are unsigned, and those prefixed with \code{i} are signed.

\specsubsubitem
The width of \terminal{int} and \terminal{uint} are implementation-defined.
\terminal{int} shall be signed, and \terminal{uint} shall be unsigned. Both
types shall have the same width, which, in bits, shall be a power of two no
smaller than 32.

\specsubsubitem
The width of \terminal{size} is implementation-defined. It shall be unsigned and
shall be able to represent the maximum length of an array type. The width in
bits shall be a power of two.

\specsubsubitem
The width of \terminal{uintptr} is implementation-defined. It shall be able to
represent the value of any \nonterminal{pointer-type} (including \textit{null}).
It shall be unsigned.

\specsubsubitem
The alignment of integer types shall be equal to their size in octets.

\specsubsubitem
The default value of an integer type shall be zero.

\informative{The following table is informative.}

\begin{tabular}{r | l l l}
Type & Width in bits & Minimum value & Maximum value \\
\hline
\terminal{i8} & 8 & -128 & 127 \\
\terminal{i16} & 16 & -32768 & 32767 \\
\terminal{i32} & 32 & -2147483648 & 2147483647 \\
\terminal{i64} & 64 & -9223372036854775808 & 9223372036854775807 \\
\terminal{u8} & 8 & 0 & 255 \\
\terminal{u16} & 16 & 0 & 65535 \\
\terminal{u32} & 32 & 0 & 4294967295 \\
\terminal{u64} & 64 & 0 & 18446744073709551615 \\
\terminal{int} & $\ge32$ & $\leq-2147483648$ & $\geq2147483647$ \\
\terminal{uint} & $\ge32$ & 0 & $\geq4294967295$ \\
\terminal{size} & $\ast$ & 0 & $\ast$ \\
\terminal{uintptr} & $\ast$ & 0 & $\ast$ \\
\end{tabular}

$\ast$ implementation-defined

\specsubsection{Floating point types}

\begin{grammar}
\nonterminaldef{floating-type} \\
	\terminal{f32} \\
	\terminal{f64} \\
\end{grammar}

\specsubsubitem
Floating point types shall represent approximations to real numbers. Floating
point types are considered \textit{numeric types}.

\specsubsubitem
The bit layout of floating point types shall be implementation-defined.

\specsubsubitem
\terminal{f32} shall have a \textit{width} of 32 and a size and alignment of 4.
\terminal{f64} shall have a width of 64 and a size and alignment of 8.

\specsubsubitem
The alignment of floating point types shall be equal to their size in octets.

\specsubsubitem
If an operation on a floating point type would cause the result to overflow or
underflow, it is truncated towards zero.

\specsubsubitem
The default value of a floating point type shall be positive zero.

\specsubsection{Rune types}

\specsubsubitem
The \terminal{rune} type represents a Unicode codepoint, encoded as a
\nonterminal{u32}.

\specsubsection{Flexible types}

\specsubsubitem
Flexible types are used during the translation phase as the result types of
literals whose final result types have not yet been determined. The three
classes of flexible type are flexible integers, flexible floats, and flexible
runes. The size and alignment of flexible types shall be undefined. Flexible
integers and flexible floats are considered numeric types.

\specsubsubitem
A flexible integer shall have a maximum and minimum value associated with it,
each of which may assume any value in the inclusive range -9223372036854775808
to 18446744073709551615. The minimum value shall represent the value of the
smallest integer literal which may have this result type, and the maximum value
shall represent the largest. The minimum value shall always be less than or
equal to the maximum, and the type shall be considered unsigned if and only if
the minimum value is greater than or equal to zero.

\specsubsubitem
The \textit{default type} for a flexible float shall be \terminal{f64}. The
default type for a flexible rune shall be \terminal{rune}. The default type for
a flexible integer shall be \terminal{int} if the maximum and minimum values are
representable as \terminal{int}, otherwise \terminal{i64} if the maximum and
minimum values are representable as \terminal{i64}, otherwise \terminal{u64} if
the maximum and minimum values are representable as \terminal{u64}, otherwise
the translation environment shall print a diagnostic message and abort.

\specsubsubitem
A flexible type may be \textit{lowered} to a different type by the
\secref{Flexible type promotion algorithm}, causing all instances of that
type within the translation environment to be replaced by the new type. If a
flexible type hasn't been lowered by the end of the translation phase, it shall
be lowered to its default type.

\specsubsubitem
There is no grammar for defining flexible types. They are for internal use only,
as the result type of \nonterminal{integer-literal},
\nonterminal{floating-literal}, and \nonterminal{rune-literal}.

\specsubsubitem
Each literal with a flexible result shall have a distinct type.

\specsubsubitem
If a flexible type is embedded within another type, it shall first be lowered to
its default type.

\informative{This is intended to ease implementation, in order to prevent a
type's ID, size, and alignment from changing partway through the translation
phase.}

\specsubsection{Other primitive types}

\textbf{The bool type}

\specsubsubitem
The \terminal{bool} type represents a boolean value, which may have one of two
states: true (represented as one) or false (represented as zero).

\specsubsubitem
The boolean type shall have the same representation, size, and alignment as
\terminal{u8}.

\specsubsubitem
If a boolean object has a value which isn't true or false, the interpretation of
the value is unspecified.

\specsubsubitem
The default value of a boolean type is false.

\textbf{The never type}

\specsubsubitem
The \terminal{never} type has no representable values, and no storage. It is
used as the result type of expressions which are guaranteed to \textit{never}
return to their caller.

\specsubsubitem
The size and alignment of \terminal{never} shall be undefined.

\specsubsubitem
If an expression whose result type is \terminal{never} returns to its caller,
behavior is undefined.

\informative{This isn't possible under normal circumstances, but conditions such
as incorrect external function implementations and invalid function pointer
casts can cause this to occur.}

\textbf{The null type}

\specsubsubitem
The null type shall have the same representation as a pointer and can only store
a specific, implementation-defined value (the \textit{null} value).

\specsubsubitem
There is no grammar for defining a value of type null, or a sub-type of null. It
is for internal use only, as the result type of the \terminal{null} expression.

\textbf{The opaque type}

The \terminal{opaque} type represents an object whose size, alignment, and
representation are unknown. As such, the opaque type has an undefined size and
alignment, and can't store any value.

\informative{No value exists whose type is \terminal{opaque}. Rather,
\terminal{opaque} is intended to be used as the secondary type of a pointer or
slice type, to exchange data whose representation is generic or unknown.}

\textbf{The valist type}

\specsubsubitem
The \terminal{valist} type is provided for compatibility with the C programming
language as specified by ISO/IEC 9899. Implementation support is optional:
implementations which do not provide C ABI compatibility must parse this type,
print a diagnostic message, and abort.

\specsubsubitem
The size and alignment of \terminal{valist} shall be an implementation-defined
non-zero multiple of eight.

\specsubsubitem
Assigning to a \terminal{valist} object from an \nonterminal{object-selector}
shall initialize a copy of the \nonterminal{object-selector}, exactly as though
a new \terminal{valist} were initialized with \terminal{vastart}, followed by
the same sequence of \terminal{vaarg} uses that had been previously used to
reach the present state of the \nonterminal{object-selector}.

\informative{If any previously evaluated \terminal{vaarg} expression on the
\nonterminal{object-selector} violated a runtime constraint, behavior of all
further \terminal{vaarg} expressions on the newly created object is undefined.}

Forward references: \subsecref{Variadic expressions}

\specsubsubitem
A \terminal{valist} object shall only be valid within the function it was
initialized in.

\textbf{The void type}

\specsubsubitem
The \terminal{void} type represents an object with no storage. Only one value
with this type exists (the \textit{void} value); this value is also the default
value.

\specsubsubitem
The size and alignment of \terminal{void} shall be zero.

\specsubsection{Pointer types}

\begin{grammar}
\nonterminaldef{pointer-type} \\
	\terminal{*} \nonterminal{type} \\
	\terminal{nullable} \terminal{*} \nonterminal{type} \\
\end{grammar}

\specsubsubitem
A pointer type is an indirect reference to an object of a secondary type. The
notation of a pointer type is a \terminal{*} prefix before the secondary type.

\specsubsubitem
A pointer type prefixed with \terminal{nullable} is considered a
\textit{nullable pointer type}, and shall refer to either a valid secondary object
or to a special value called \textit{null}. A non-nullable pointer type shall
\textbf{always} refer to a valid secondary object.

\specsubsubitem
A pointer type's secondary type must have a non-zero size. Its size may be
undefined. The secondary type shall not be \terminal{never}.

\specsubsubitem
The representation of a pointer type shall be implementation-defined, and it
shall have an implementation-defined size and alignment.

\specsubsubitem
The default value of a nullable pointer type is null. The default value of a
non-nullable pointer type is undefined.

\specsubsubitem
A pointer type shall be equivalent to another pointer type only if they share
the same secondary type and nullable status.

\specsubsection{Struct and union types}

\begin{grammar}
\nonterminaldef{struct-union-type} \\
	\terminal{struct} \optional{\terminal{@packed}} \terminal{\{} \nonterminal{struct-union-fields} \terminal{\}} \\
	\terminal{union} \terminal{\{} \nonterminal{struct-union-fields} \terminal{\}} \\

\nonterminaldef{struct-union-fields} \\
	\nonterminal{struct-union-field} \optional{\terminal{,}} \\
	\nonterminal{struct-union-field} \terminal{,} \nonterminal{struct-union-fields} \\

\nonterminaldef{struct-union-field} \\
	\optional{\nonterminal{offset-specifier}} \nonterminal{name} \terminal{:} \nonterminal{type} \\
	\optional{\nonterminal{offset-specifier}} \nonterminal{struct-union-type} \\
	\optional{\nonterminal{offset-specifier}} \nonterminal{identifier} \\

\nonterminaldef{offset-specifier} \\
	\terminal{@offset} \terminal{(} \nonterminal{expression} \terminal{)}
\end{grammar}

\specsubsubitem
The \textit{struct type} and \textit{union type} \textit{collect} multiple
types, name them, and assign them \textit{offset}s within their storage area. A
union type stores all of its values at the same offset; a struct type may store
its values at different offsets. A type defined with the \terminal{struct}
terminal is a struct type and uses the struct type class; if the
\terminal{union} terminal is used the type is a union type with the union type
class.

\specsubsubitem
The \nonterminal{struct-union-fields} list denotes, in order, the subvalues
which are collected by a struct or union, and potentially assigns a
\nonterminal{name} to each.

\specsubsubitem
For a struct type, the offset of each field is equal to the minimum
\textit{aligned} offset which would meet the alignment requirements of the
field's type and which is greater than the offset of the previous field plus the
size of the previous field. The implementation shall add \textit{padding} to
meet the alignment requirements of struct fields. For a union type, the offset
of all members is zero. Padding shall additionally be added to the end of a
struct or union type whose alignment is non-zero, such that the total size of
the struct or union type modulo its alignment is zero. A struct or union type's
alignment is the maximum alignment among its fields.

\specsubsubitem
For a struct type using the \terminal{@packed} modifier, the offset of each
field shall be computed without respect to alignment, such that each field's
offset is equal to the offset of the previous field plus the size of the
previous field. No additional padding shall be added to the end of the struct
type in this case. If the alignment of the struct fields or the struct type
itself would not meet the alignment requirements for their respective type, the
behavior is implementation-defined. The implementation shall either raise a
diagnostic message and terminate the translation phase, or shall support
unaligned memory accesses (perhaps at a cost to performance).

\specsubsubitem
The type of each struct or union field shall have a definite size.

\specsubsubitem
If given, the \nonterminal{offset-specifier} shall override the computed offset
for a given field. If the user-defined offset for a field would not meet the
alignment requirements for that type, the behavior is implementation-defined.
The implementation shall either raise a diagnostic message and terminate the
translation phase, or shall support unaligned memory accesses (perhaps at a
cost to performance).

\specsubsubitem
The \nonterminal{expression} given for the \nonterminal{offset-specifier} shall
be limited to the \secref{Translation compatible expression subset}, and
shall have an integer type and a positive or zero value.

\specsubsubitem
The \nonterminal{offset-specifier} shall not be given for a \terminal{union}
type.

\specsubsubitem
The default value of a struct type shall be defined as a value whose fields
assume the default values of their respective types. If any field's default
value is undefined, the struct type's default value shall also be undefined.

\specsubsubitem
The default value of a union type shall be undefined.

\specsubsubitem
If the \nonterminal{struct-union-type} form of \nonterminal{struct-union-field}
is given, the parent type shall collect the fields of the child type as its
own. The offset of each field within the child type shall be the sum of the
offset within the child type and the offset the child type occupies within the
parent struct. The \nonterminal{identifier} form shall be interpreted in the same
manner as a \nonterminal{struct-union-type} if it refers to a type alias of a
struct or union type, or an alias thereof, otherwise a diagnostic message shall
be printed and the translation phase shall abort.

Forward references: \subsecref{Type aliases}

\specsubsubitem
A struct or union type shall be equivalent to another struct or union type if
their fields are of equivalent name, type, and offset, without respect to the
order of their appearance in the program source.

\informative{The following types are equivalent: \\
\code{struct \{ a: int, b: int \}} \\
\code{struct \{ a: int, struct \{ b: int \} \}}}

\specsubsubitem
Each field name (including names of embedded fields) shall be unique within the
set of all field names of the \nonterminal{struct-union-type}.

\specsubsection{Tuple types}

\begin{grammar}
\nonterminaldef{tuple-type} \\
	\terminal{(} \nonterminal{tuple-types} \terminal{)} \\

\nonterminaldef{tuple-types} \\
	\nonterminal{type} \terminal{,} \nonterminal{type} \optional{\terminal{,}}\\
	\nonterminal{type} \terminal{,} \nonterminal{tuple-types} \\
\end{grammar}

\specsubsubitem
A tuple type stores two or more values of arbitrary types in a specific order.
It is similar to a struct type, but without names for each of its subvalues.
Each value is stored at a given offset, possibly with padding added to meet
alignment requirements.

\specsubsubitem
The offset of each value is equal to the minimum \textit{aligned} offset which
would meet the alignment requirements of the value's type and which is greater
than the offset of the previous value plus the size of the previous value type.
The implementation shall add \textit{padding} to meet the alignment
requirements of tuple values. Padding shall additionally be added to the end of
a tuple type whose alignment is non-zero, such that the total size of the tuple
type modulo its alignment is zero.

\specsubsubitem
The size of a tuple is the sum of the sizes of its value types plus any necessary
padding. The alignment is the maximum alignment among its value types.

\specsubsubitem
The type of each tuple value shall have a definite size.

\specsubsubitem
The default value of a tuple type shall be defined such that its values assume
the default values of their respective types. If any of the subtypes do not
have a default value, neither does the tuple type.

\specsubsubitem
Two tuple types shall be equivalent to each other if they have the same value
types in the same order.

\specsubsection{Tagged union types}

\begin{grammar}
\nonterminaldef{tagged-union-type} \\
	\terminal{(} \nonterminal{tagged-types} \terminal{)} \\

\nonterminaldef{tagged-types} \\
	\nonterminal{type} \terminal{|} \nonterminal{type} \optional{\terminal{|}}\\
	\nonterminal{type} \terminal{|} \nonterminal{tagged-types} \\
\end{grammar}

\specsubsubitem
A tagged union stores a value of \textbf{one} of its constituent types, as well
as a \textit{tag} which indicates which of the constituent types is selected.
The constituent types are defined by \nonterminal{tagged-types}.

\specsubsubitem
The representation of a tagged union shall be a \terminal{u32}, in which the
tag value is stored, followed by sufficient space to store any of the possible
constituent types. Padding shall be inserted prior to the tag if necessary to
meet the maximum alignment among the tagged union members and the \terminal{u32}
field.

% XXX: This LaTeX hackery is garbage

\informative{The representation of \code{(u8 | i16)} is equivalent to}
\begin{codesample}
struct {
	tag: u32,
	union {
		unsigned: u8,
		signed: i16,
	},
}
\end{codesample}
\informative{And the representation of \code{(u64 | []u8)} is equivalent to}
\begin{codesample}
struct {
	pad: u32,
	tag: u32,
	union {
		unsigned: u64,
		slice: []u8,
	},
}
\end{codesample}

\specsubsubitem
The tag value shall be the type ID of the type which is selected from the
constituent types. This value shall be stored at the \terminal{u32} field and
shall indicate which type is stored in the value area.

\specsubsubitem
The alignment of a tagged union type shall be the alignment of the
\terminal{u32} type or the maximum alignment of the constituent types,
whichever is greater.

\specsubsubitem
The size of a tagged union type shall be the maximum size of its constituent
types, plus the size of the \terminal{u32} type, plus any padding added per
\subsubitemref{Tagged union types}{2}.

\specsubsubitem
If a member type among \nonterminal{tagged-types} is a tagged union type, it
shall be reduced such that nested tagged union type is replaced with its
constituent types in the parent union.

\informative{The types \code{(A | (B | (C | D)))} and
\code{(A | B | C | D)} are equivalent.}

\specsubsubitem
The default value of a tagged union type is undefined.

\specsubsubitem
A tagged union type shall be equivalent to another tagged union type if they
share the same set of secondary types, without regard to order, and considering
the secondary types of nested tagged unions as members of the set of their
parent's secondary types.

\informative{It follows that the types \code{(A | B)} and \code{(B | A)}
are equivalent.}

\specsubsection{Slice and array types}

\begin{grammar}
\nonterminaldef{slice-array-type} \\
	\terminal{[} \terminal{]} \nonterminal{type} \\
	\terminal{[} \nonterminal{expression} \terminal{]} \nonterminal{type} \\
	\terminal{[} \terminal{*} \terminal{]} \nonterminal{type} \\
	\terminal{[} \terminal{\_} \terminal{]} \nonterminal{type} \\
\end{grammar}

\specsubsubitem
An \textit{array type} stores one or more items of a uniform secondary type.
The number of items stored in an array type (its \textit{length}) is an
attribute of the array type and is specified during the translation phase. The
secondary type shall have a definite, non-zero size.

\specsubsubitem
The \nonterminal{expression} representation is used for array types of a
determinate length, that is, with a determinate number of items. Such arrays
are \textit{bounded}. The \nonterminal{expression} must evaluate to a positive
integer value, and shall be limited to the \secref{Translation compatible
expression subset}.

\specsubsubitem
An array type may be \textit{unbounded}, in which case the length is not known.
The \terminal{*} representation indicates an array of this type.

\specsubsubitem
An array may be bounded, but infer its length from context, using the
\terminal{\_} representation. Such an array is said to be
\textit{context-defined}.

\specsubsubitem
An array type may be \textit{expandable}. This state is not represented in the
type grammar, and is only used in specific situations. Array types are presumed
to be non-expandable unless otherwise specified.

\specsubsubitem
The representation of an \textit{array type} shall be the items concatenated one
after another, such that the offset of the \textit{N}th item (starting at zero)
is determined by the equation $N \times S$, where \textit{S} is the size of the
secondary type.

\specsubsubitem
A \textit{slice type} stores a pointer to an unbounded array type, with a given
\textit{capacity}, and \textit{length}, which respectively refer to the number
of items that the unbounded array \textbf{may} store without re-allocation, and
the number of items which are \textbf{currently valid}. The secondary type shall
not have a size of zero, and shall not be \terminal{never}. The representation
with no lexical elements between \terminal{[} and \terminal{]} indicates a slice
type.

\specsubsubitem
The representation of a slice type shall be equivalent to the following struct
type:

\begin{codesample}
struct {
	data: nullable *opaque, // See notes
	length: size,
	capacity: size,
}
\end{codesample}

If the secondary type has a definite size, the type of the \textit{data} field
shall be a nullable pointer to an unbounded array of the secondary type,
otherwise, it shall be a nullable pointer to an unbounded array whose secondary
type is unknown.

\specsubsubitem
The alignment of an array type shall be equivalent to the alignment of the
underlying type. The alignment of a slice type shall be equivalent to the
alignment of the \terminal{size} type or \subsecref{Pointer types}, whichever is
greater.

\specsubsubitem
The size of a bounded array type shall be equal to $N \times S$, where N is the
length and S is the size of the underlying type. The size of an unbounded array
is undefined. The size of a slice type shall be equal to the size of the struct
type defined by \subsubitemref{Slice and array types}{6}.

\specsubsubitem
The default value of a bounded array type shall be an array whose
members are all set to the default value of the secondary type. If the array is
unbounded, or if the default value of the secondary type is undefined, the
default value of the array type is undefined.

\specsubsubitem
The default value of a slice type shall have the capacity and length fields set
to zero and the data field set to null.

\specsubsubitem
An array type shall be equivalent to another array type only if its length and
secondary types are equivalent. A slice type shall only be equivalent to a
slice type with the same secondary type.

\specsubsection{String types}

\specsubsubitem
The \terminal{str} type (or \textit{string type}) stores a reference to a
sequence of Unicode codepoints, encoded as \hbox{UTF-8}, along with its
\textit{length} and \textit{capacity}. The length and capacity are measured in
octets, rather than codepoints.

\specsubsubitem
The representation of the string type shall be equivalent to the following
struct type:

\begin{codesample}
struct {
	data: nullable *[*]const u8,
	length: size,
	capacity: size,
}
\end{codesample}

\specsubsubitem
The default value of a string type shall have the length field set to zero and
the data field set to null.

\specsubsection{Function types}

\begin{grammar}
\nonterminaldef{function-type} \\
	\terminal{fn} \nonterminal{prototype} \\

\nonterminaldef{prototype} \\
	\terminal{(} \optional{\nonterminal{parameter-list}} \terminal{)} \nonterminal{type} \\

\nonterminaldef{parameter-list} \\
	\nonterminal{parameters} \optional{\terminal{,}} \\
	\nonterminal{parameters} \terminal{...} \\
	\nonterminal{parameters} \terminal{,} \terminal{...} \\
	\terminal{...} \\

\nonterminaldef{parameters} \\
	\nonterminal{parameter} \\
	\nonterminal{parameters} \terminal{,} \nonterminal{parameter} \\

\nonterminaldef{parameter} \\
	\nonterminal{name} \terminal{:} \nonterminal{type} \\
	\nonterminal{type} \\
\end{grammar}

\specsubsubitem
Function types represent a procedure which may be completed in the
\secref{Execution environment} to obtain a result and possibly cause side
effects (see \subsubitemref{Program execution}{1}).

\specsubsubitem
The attributes of a function type are its \textit{result type} and
\textit{parameters}. A function type must have one result type, and zero or more
parameters. Within a list of \nonterminal{parameters}, no two parameters which
use the \nonterminal{name} form may have the same name. The type of every
parameter shall have a definite size.

\specsubsubitem
If the second form of \nonterminal{parameter-list} is used, the final parameter of
the function type uses \textit{Hare-style variadism}. If the third or fourth
form is used, the function uses \textit{C-style variadism}. The variadism of a
function type affects the calling semantics for that function.

\begin{codesample}
// Hare-style variadism:
fn(x: int, y: int, z: int...)

// C-style variadism:
fn(x: int, y: int, ...)
\end{codesample}

Forward references: \subsecref{Calls}

\specsubsubitem
The implementation is not required to support C-style variadism. If the
implementation does not support C-style variadism, it must print a diagnostic
message and abort the translation environment for programs which attempt to
utilize it.

\specsubsubitem
The type of a parameter which uses Hare-style variadism shall be a slice of the
specified type.

\informative{Therefore, in the case of \code{fn(x: int...)}, the type of x
shall be \code{const []int}.}

\specsubsubitem
The size, alignment, default value, and storage semantics of function types is
undefined.

\specsubsubitem
The function's result type, list of parameter types (in order), and its
variadism, are distinct characteristics of the function type, for the purpose of
determining equivalency.

\specsubsection{Type aliases}

\begin{grammar}
\nonterminaldef{alias-type} \\
	\nonterminal{identifier} \\

\nonterminaldef{unwrapped-alias} \\
	\terminal{...} \nonterminal{identifier} \\
\end{grammar}

\specsubsubitem
A type alias assigns an \nonterminal{identifier} a unique type which is an
alias for another type or a name for a set of enum values.

\informative{The grammar for an \nonterminal{alias-type} does not specify the
underlying type. The underlying type is specified at the time it is declared,
see \secref{Declarations}.}

\specsubsubitem
A type alias shall have the same storage, alignment, size, and default value as
its underlying type.

\specsubsubitem
The underlying type of a type alias shall not be \terminal{never}.

\specsubsubitem
A type alias that represents an enum type shall have a default value of zero
only if one of the enum values is equal to zero, otherwise its default value is
undefined. A type alias that doesn't represent an enum type shall have the same
default value as its underlying type.

\specsubsubitem
Each type alias (uniquely identified by its \nonterminal{identifier}) shall be
a unique type, even if it shares its underlying type with another type alias.

\specsubsubitem
An \nonterminal{unwrapped-alias} shall refer to the underlying type of the given
type alias, rather than the type alias itself.

\informative{This notably affects the relationship between type aliases and
tagged unions. In the following example, union\_a and union\_b have different
storage semantics, the former being a tagged union of two other tagged unions,
and the latter being reduced to a single tagged union.}

\begin{codesample}
type signed = (i8 | i16 | i32 | i64 | int);
type unsigned = (u8 | u16 | u32 | u64 | uint);
type union_a = (signed | unsigned);
type union_b = (...signed | ...unsigned);
\end{codesample}
